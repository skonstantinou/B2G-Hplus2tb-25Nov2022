\Slide{t}{
Many BSM theories need to enlarge their Higgs sector to two Higgs doublets
\begin{itemize}%1
  % https://arxiv.org/pdf/1706.07414.pdf (page 11)
\item The minimal \acp{2HDM} predict 5 physical states:
  \begin{itemize}%2
    \item two neutral, \CP-even particles \Ph\ and \PH\ ( $\mh \leq \mH$)
    \item one neutral, \CP-odd particle \PSAz
    \item two charged Higgs bosons \PHpm
  \end{itemize}%2
\end{itemize}%1

\vspace{0.35cm}

\acs{SM} fermion coupling to 2HDs (no \acsp{FCNC}):
\twoColumnsAsymAlt
    {
      \small
      \begin{itemize}
      \item[\alertA{I}] \alertA{All quarks \& leptons couple to $\Phi_{2}$}
        \vspace{-0.08cm}
      \item[\alertB{II}] \alertB{All $u$-type to $\Phi_{2}$ and all $d$-type \& $\ell$ to $\Phi_{1}$} %\NeutralTag{MSSM-like}
        \vspace{-0.08cm}
      \item[\alertC{X}] \alertC{Both $u$ \& $d$ types couple to $\Phi_{2}$, all $\ell$ to $\Phi_{1}$}
        \vspace{-0.08cm}
      \item[\alertD{Y}] \alertD{Roles of two doublets reversed wrt \Type{II}}
      \end{itemize}
    }
    {
      \vspace{-0.25cm}
      
      \resizebox{0.87\linewidth}{!}{
        % https://arxiv.org/pdf/1607.01320.pdf
%\note{The most popular models of the Yukawa interactions in the 2HDM 
%  (also referred to as ``Types''). The symbols $u$, $d$, $\ell$ refer to up-
%  and down-type quarks, and charged leptons of any generation.
%  Here, $\Phi_1$ and $\Phi_2$ refer to the Higgs doublet coupled to
%  the particular fermion.}
\resizebox{0.65\linewidth}{!}{
  \begin{tabular}{l c c c}
    \hline
    Type & $u$ & $d$ & $\ell$ \\
    \hline
    \rowcolor{\ColourA}
    I & $\Phi_2$ & $\Phi_2$ & $\Phi_2$
    \\%[0.1cm]
    \rowcolor{\ColourB}
    II & $\Phi_2$ & $\Phi_1$ & $\Phi_1$ %\NeutralTag{MSSM}\\
    \\
    \rowcolor{\ColourC}
    III (X) & $\Phi_2$ & $\Phi_2$ & $\Phi_1$
    \\%[0.1cm]
    \rowcolor{\ColourD}
    IV (Y)  & $\Phi_2$ & $\Phi_1$ & $\Phi_2$
    \\%[0.1cm]
    \hline
  \end{tabular}}


      }
    }

\vspace{0.45cm}
    
For each \acsp{2HDM} type there are 7 free parameters (incl. \mh, \mH, \mA, \mHpm):
\begin{enumerate}
  \setcounter{enumi}{4}
\item \tanbetaDef, the ratio of the Higgs doublet \acsp{VEV}
\item \sinba, \mixingAngle: the mixing angle of the \CP-even states
\item \massMatrixDiagonal, diagonal term of the mass matrix of the Higgs doublets

\end{enumerate}
}

\Slide{t}{
Three mass categories are commonly defined in \PHpm\ searches:
\begin{itemize}
\item \alertA{Light $\mHpm < \mtop - \mbottom$ },
  \alertB{intermediate\tikzmark{f0} $\mHpm \sim \mtop$},
  \alertC{heavy $\mHpm > \mtop + \mbottom$}
\end{itemize}

\vspace{0.2cm}               
Decay \acsp{BR} model-dependent $\Rightarrow$ different searches constrain different scenarios.

% https://arxiv.org/pdf/2005.08900.pdf
\twoFigColumns
{LHCHXSWG_v1/LowTanBetaM50to200/mh125_brs_tanb0p5}{$M_{h}^{125}$,
  \tanbeta = 0.5}
{LHCHXSWG_v1/LowTanBetaM90to1000/mh125_br_tb}{ $M_{h}^{125}$, \PHpm $\rightarrow$ tb}
{0.8}

\vspace{0.2cm}

\textbf{\acsp{BR} of \textcolor{kGreen}{\PHpm $\rightarrow$ tb}
  dominanates at high $\mHpm$, for wide range of \tanbeta}
}

\Slide{t}{

  \vspace{0.1cm}
  
This analysis searches for a heavy \PHpm

\vspace{-0.1cm}

\threeFigColumnsCustomSize
    {paper/feynman/hplus_4FS}{4FS}
    {paper/feynman/hplus_5FS}{5FS}
    {paper/feynman/hplus_sChannel}{s-channel}
    {0.85}

    \vspace{0.5cm}
    
    \textbf{\textcolor{kBlue}{Fully-hadronic} final state of associated production characterised by}:

  \twoColumns
      {

        \vspace{0.25cm}
        
        \begin{itemize}
          \small
        \item High jet \& \bjet\ multiplicities 
        \item[\Good] Large branching ratio $\BR\simeq46\%$
        %\item Absence of neutrinos $\Rightarrow$ low \ptmiss
        \item[\Good] Invariant mass reconstruction of \PHpm
        \item[\Bad] QCD multijet \& \ttbar background % irreducible
        \item[\Bad] Combinatorial (self-)background
        \end{itemize}
      }
      {        
        \vspace{-0.25cm}
        {\oneFig{tikz/pdf/HPlus4FS_HToTB_FH}{0.9}}
      }
}

\Slide{t}{
       Various $\mathrm{m_{H^\pm}}$ reconstruction techniques
       available due to signal process kinematics:

       \vspace{0.1cm}       

    \begin{itemize}
      \small
    \item\textcolor{kDarkGreen}{\textbf{Resolved \PQt}:} At moderate
      \mHpm\ \& \ptHpm the decay products of \PHpm are well separated
    \item \textcolor{kBlue}{\textbf{Boosted \PW/\PQt\tikzmark{bW1}}:}
      As $\mathrm{m_{H^\pm}}$ increases the \PHpm\ decay products
      become boosted
    \item \textcolor{kOrange}{\textbf{Boosted \Hpm}:} As \ptHpm
      increases its decay products become collinear \tikzmark{bH1}      
    %\item As the $\mathrm{p_{T,H^{\pm}}}$ increases, its decay
      %products become collinear
    \end{itemize}

    \vspace{0.1cm}
    
  \begin{columns}[T]
      \begin{column}{0.20\textwidth}
        \begin{tcolorbox}[transparentStyle, title=\tColorBoxTitle{\tikzmark{rTop2}\vspace{-0.15cm}\resolvedTop}]
          \oneFig{tikz/pdf/h2tb_ResolvedTop}{1.0}
        \end{tcolorbox}
      \end{column}
      \begin{column}{0.20\textwidth}
        \begin{tcolorbox}[transparentStyle, title=\tColorBoxTitle{\tikzmark{bW2}\vspace{-0.15cm}\boostedW}]
        \oneFig{tikz/pdf/h2tb_BoostedW}{1.0}
        \end{tcolorbox}
      \end{column}
      \begin{column}{0.20\textwidth}
        \begin{tcolorbox}[transparentStyle,  opacitytext=1.0, title=\tColorBoxTitle{\tikzmark{bTop}\vspace{-0.15cm}\boostedTop}]
        \oneFig{tikz/pdf/h2tb_BoostedTop}{1.0}
        \end{tcolorbox}
      \end{column}
       \begin{column}{0.20\textwidth}
         \begin{tcolorbox}[transparentStyle, title=\tColorBoxTitle{boosted \PHpm}\vspace{-0.15cm}\tikzmark{bH2}]
         \oneFig{tikz/pdf/h2tb_Boostedhiggs}{1.0}
          \end{tcolorbox}
      \end{column}
      \begin{column}{0.20\textwidth}
        \begin{tcolorbox}[transparentStyle, title=\tikzmark{bH3}\tColorBoxTitle{\vspace{-0.15cm}boosted \PHpm}]
          \oneFig{tikz/pdf/h2tb_BoostedhiggsLargePt}{1.0} 
        \end{tcolorbox}
      \end{column}
  \end{columns}

  \vspace{0.35cm}
  
  \twoColumns
      {
        \small
        \hspace{1.5cm} \textbf{Previous results}
        \vspace{0.1cm}
        \CustomiseLineSpacing{0.95}
        \begin{itemize}
        \item \textcolor{kDarkGreen}{\textbf{Resolved \PQt}},
          \textcolor{kBlue}{\textbf{Boosted \PW/\PQt}}\newline studied separately by dedicated analyses
        \item 2016 ReReco data
        \item CADI \href{https://cms.cern.ch/iCMS/analysisadmin/cadilines?line=HIG-18-015}{\textcolor{kBlue}{HIG-18-015}}
        \end{itemize}
      }
      {
        \small
        \hspace{2cm}  \textbf{This work}
        \vspace{0.1cm}
        \CustomiseLineSpacing{0.95}
        \begin{itemize}          
          \item \textcolor{kDarkGreen}{\textbf{Resolved \PQt}}, \textcolor{kBlue}{\textbf{Boosted \PQt}}
          \item Full Run II data
          \item This talk: status of 2017-2018 data
            %\item Last progress report in HExtended meeting:
          \item Last report (HExtended):
            \href{https://indico.cern.ch/event/1071752/contributions/4578521/attachments/2333742/3977531/HiggsExtended_MKolosova_25Oct2021.pdf}{\textcolor{kBlue}{25
            Oct 2021}}
        \end{itemize}        
      }
}

\MySection{Previous results}
\Slide{t}{

  
  \hspace{4.0cm} \textcolor{kDarkGreen}{\textbf{Resolved}} (UCY, HIP)

  \vspace{0.1cm}

  %\CustomiseLineSpacing{1.1}  
  \begin{itemize}
    \small
  \item Resolved t ($t^{res}$) identification: custom top tagger (BDT)
  \item Selected events contain $\ge$ 7 jets, $\ge$ 3 b-tagged, 2 $t^{res}$
  \item \PHpm mass reconstruction ($m_{bt}$): leading \pT $t^{res}$ +
    leading \pT b jet
  \item Main background:
    \begin{itemize}
      \footnotesize
    \item Misid. B: From data using CRs (ABCD method)
    \item Genuine B: from simulation
    \end{itemize}
  \item $m_{bt}$ is used to extract the signal in the presence of the SM background.
  \end{itemize}

  \vspace{-0.2cm}
  
  \twoFigColumns
      {CMS-HIG-18-015/Figure_003-a.pdf}{$t^{res}$ efficiency}
      %{topEffic_Resolved_HIG18015}{$t^{res}$ efficiency}
      {CMS-HIG-18-015/Figure_005-b.pdf}{$m_{bt}$ Post-fit}
      {0.55}
}

\Slide{t}{

  \hspace{4.0cm} \textcolor{kBlue}{\textbf{Boosted}} (MIT, BUAP)  

  \vspace{0.05cm}  

  \begin{itemize}
    \small
  \item Events are split in four main categories
  \end{itemize}

  \vspace{-0.4cm}
  
  \begin{columns}[T]
      \begin{column}{0.145\textwidth}
        \begin{tcolorbox}[transparentStyle,
            title=\tColorBoxTitle{\vspace{-0.15cm} t0b}]
          \oneFig{tikz/pdf/h2tb_HIG-18-015_t0b_2bjets}{1.0}
        \end{tcolorbox}
      \end{column}
      \begin{column}{0.145\textwidth}
        \begin{tcolorbox}[transparentStyle,
            title=\tColorBoxTitle{\vspace{-0.15cm} t1b}]
        \oneFig{tikz/pdf/h2tb_HIG-18-015_t1b_2bjets}{1.0}
        \end{tcolorbox}
      \end{column}
      \begin{column}{0.145\textwidth}
        \begin{tcolorbox}[transparentStyle,  opacitytext=1.0,
            title=\tColorBoxTitle{\vspace{-0.15cm} Wbb}]
        \oneFig{tikz/pdf/h2tb_HIG-18-015_Wbb_2bjets}{1.0}
        \end{tcolorbox}
      \end{column}
       \begin{column}{0.145\textwidth}
         \begin{tcolorbox}[transparentStyle,
             title=\tColorBoxTitle{\vspace{-0.15cm} Wbj}]
         \oneFig{tikz/pdf/h2tb_HIG-18-015_Wbj_2bjets}{1.0}
          \end{tcolorbox}
      \end{column}
  \end{columns}

  \vspace{-0.2cm}
  %\oneFigColumns{boosted_categories.png}{}{0.85}

  \begin{itemize}
    \small
  \item Boosted $t$/$W$ identification: Based on \mSD[],
    \subjettiness[]{N}, $N_{b \ \text{subjets}}$
  \item \PHpm mass reconstruction ($m_{bt}$): $t$ + leading \pT b jet
  \item Further categorization according to:
    \begin{itemize}
      \setlength{\itemindent}{1mm}
      \footnotesize
    \item $N_{b}$ $\in [=1, =2, \geq 3]$
    \item $N_{j}^{extra}$ $\in [< 3, \geq 3]$
    \item \mHpmReco $\in [\text{below}, \text{in}, \text{above}]$ of FWHM of signal
    %\item Number of bjets (=1, =2, $\ge$3)
    %\item Number of jets not used in \PHpm (<3, $\ge$ 3)
    %\item Below, in and above the $m_{bt}$ window
    \end{itemize}
  \end{itemize}

  %\vspace{-0.1cm}

  \twoColumns
      {
        \begin{itemize}
          \small
        \item Main background
          \begin{itemize}
            \footnotesize
          \item[QCD]: from data using CRs (inverted
            \subjettiness[]{N}), sidebands with \mHpmReco
            $\in [\text{below},\text{above}]$)
          \item[\ttbar]: from sim., normalized in CR with 1
            $\ell$
          \end{itemize}
        \end{itemize}
      }
      {
        
        \vspace{-2.7cm}
        \oneFigColumns{CMS-HIG-18-015/Figure_005-a.pdf}{}{0.57}
      }

      \vspace{-0.1cm}
      
      \begin{itemize}
        \small
        \item $H_{T}$ is used to extract the signal from SM background inside the $m_{bt}$ window.
      \end{itemize}
}

\Slide{t}{
  
  %\threeFigColumnsBig
  %    {CMS-HIG-18-015/Figure_006-a.pdf}{}
  %    {CMS-HIG-18-015/Figure_006-b.pdf}{}
  %    {CMS-HIG-18-015/Boosted_Figure-aux_001-b.pdf}{}

  \twoColumns
      {
        \vspace{-0.2cm}
        
        \oneFigColumns{limit_assoc_HIG18015}{Associated production}{0.7}
      }
      {
        %Results are interpreted to set model independent upper limits on the
        %$\mathrm{\sigma_{H^{\pm}}}\times\mathcal{B}(\mathrm{H^{\pm}\rightarrow tb})$

        \vspace{0.75cm}

        \textbf{Resolved and Boosted analysis overlayed limits}
        \begin{itemize}
          \scriptsize
        \item Boosted: best sensitivity for $\mathrm{m_{H^\pm}}$ > 0.8 TeV
        %\item Resolved: most stringent limits at lower masses
        \item Reported limit at each mass value is determined by
          the analysis with the best expected sensitivity.
        \item 21.3 to 0.007 pb for masses 0.2 to 3 TeV
        \item No excess above the estimated background% is observed
        \item Interpretation in hMSSM: maximum \tanb = 0.88 is
          excluded for \mHpm = 0.20-0.55~TeV
        \end{itemize}

        %The hMSSM benchmark scenario assumes that the discovered Higgs boson is the light
        %Higgs boson in the 2HDM and that the SUSY particles have masses too large to be directly
        %observed at the LHC.
      }
      
  \twoColumnsAsymAlt
      {
        \twoFigColumns
            {CMS-HIG-18-015/Figure_006-b.pdf}{s-channel}
            {CMS-HIG-18-015/Boosted_Figure-aux_001-b.pdf}{Assoc. boosted categories}
            {1.}
      }
      {

        \vspace{0.75cm}
        
        \begin{itemize}
        \item Boosted analysis sets upper limits in the s-channel
          production
          \begin{itemize}
            \scriptsize
          \item 4.5-0.023 pb for \mHpm 0.8 to 3 TeV
          \end{itemize}
        \item Boosted categories
          \begin{itemize}
            \scriptsize
          \item Most sensitive main category is $t1b$
          \item Least sensitive category is $Wbj$
          \end{itemize}
        \end{itemize}
      }
  
}

\MySection{Resolved}
\MySubSection{event selection}

\Slide{t}{

  \textbf{Signal region (SR):}
  \vspace{-0.2cm}
  \begin{center}
    \resizebox{1.0\linewidth}{!}{
      \begin{tabular}{l l}
        \hline
        \cellcolor[gray]{.9}Trigger  & \cellcolor[gray]{.9}$H_{T}$ + multijet + 1 or 2 b jets\\
        $e$ veto & \pT > 10 GeV, $|\eta|$ < 2.4, Loose miniIso, cutBasedElectronID (veto) \\%\_Fall17\_94X\_V2\_veto \\
        \cellcolor[gray]{.9}$\mu$ veto & \cellcolor[gray]{.9}\pT > 10 GeV, $|\eta|$ < 2.4, Loose miniIso isCutBasedIDLoose\\
        $\tau$ veto & \pT > 20 GeV, $|\eta|$ < 2.3, DeepTau \dDeepTau[vloose]{\Pe}, \dDeepTau[medium]{\mu}, \dDeepTau[loose]{j}\\
        \cellcolor[gray]{.9}$\ge$ 7 jets & \cellcolor[gray]{.9}$p_{T}^{6th}$ > 40 GeV, $p_{T}^{7th}$ > 30 GeV, $|\eta|$ < 2.4, Tight ID, $H_{T}$ > 500 GeV\\
        $\ge$ 3 b jets & \pT > 40 GeV, DeepJet Medium WP \\
        \cellcolor[gray]{.9}$\ge$ 1 resolved top ($t^{res}$) & \cellcolor[gray]{.9}custom DNN medium, $130 < m_{\mathrm{t^{res}}} < 210~$GeV\\
        \hline
      \end{tabular}
    }
  \end{center}

  \vspace{0.75cm}
  
  
  \twoColumnsAsymAlt
      {
        \normalsize
        $\ \ $ \textbf{SR categorization based on $t^{res}$}
        \begin{itemize}
          \small
        \item \textcolor{kDarkGreen}{$1M1L_{t^{res}}$}: medium $t^{res}_{\PHpm}$ \newline
          \textcolor{white}{$1M1L_{t^{res}}$:} loose-not-medium $t^{res}_{assoc}$
        \item \textcolor{kDarkGreen}{$2M_{t^{res}}$}: both $t^{res}$ medium tagged
        \end{itemize}

        \vspace{0.3cm}

        $\ \ $ \textbf{Invariant \PHpm~mass reconstruction}:
        \begin{itemize}
          \large
        \item[] \textcolor{kRed}{\mHpmReco = $t^{res}_{ \ ldg \ p_{T}}$
          + bjet$^{free}_{ldg \ p_{T}}$}
        \end{itemize}
      }
      {
        \vspace{-0.4cm}
        \oneFigColumns{tikz/pdf/HPlus4FS_HToTB_ResolvedTop_InvMass}{}{1.0}%\mHpmReco=\rTop$_{ \ ldg \ p_{T}}$ + bjet$^{free}_{ldg \ p_{T}}$}{0.68}
      }
}

\MySubSection{top tagging}
\Slide{t}{
  \small

  \vspace{0.1cm}

  A fully connected NN is developed to reconstruct resolved top-quarks

  \begin{itemize}
  \item Distinguishes trijets from top-quark decays and trijets from combinatorial background.
  \end{itemize}

    \twoColumns
        {
          \small

          \begin{itemize}
          \item Training on simulated $t\bar{t}$ events
        \begin{itemize}
          \small
        \item \textcolor{kDarkGreen}{Signal:} truth-matched trijets
        \item \textcolor{kRed}{Background:} non-matched trijets
        \end{itemize}
         \end{itemize}
      }
      {
         \vspace{-0.2cm}
        \twoFigColumns
            {top-training/top_matched}{ \textcolor{kDarkGreen}{signal}}
            {top-training/top_unmatched1b}{\textcolor{kRed}{background}}
            {0.45}
      }

  Mass decorrelation using sample reweighting:

  \begin{itemize}
    \small
  %\item Input variables are uncorrelated to the top-quark mass distribution ($m_{top}$)
  \item \textcolor{kRed}{Background} is reweighted such that $m_{top}$ matches the \textcolor{kDarkGreen}{signal}.
  \end{itemize}

  \vspace{0.2cm}
  
  SF vs $t^{res}$ \pT measured in a region with 1 isolated $\ell$
  
  \vspace{-0.4cm}

  \threeColumns
      {\oneFigColumns{CMS-HIG-21-010_Figure_003.pdf}{}{0.9}}
      {\oneFigColumns{topTagging/trijetMass_3WPs}{}{0.9}}
      {
        \footnotesize
        \vspace{1.5cm}

        %\begin{itemize}
        %  \scriptsize
        %\item[] In \href{http://cms-results.web.cern.ch/cms-results/public-results/publications/HIG-21-010/index.html}{\textcolor{kBlue}{HIG-21-010}}
        %\item[] Documentation: \href{https://cms.cern.ch/iCMS/user/noteinfo?cmsnoteid=CMS\%20AN-2021/019}{\textcolor{kBlue}{AN 2021/019}}
        %\item[] Approved by \href{https://indico.cern.ch/event/1042288/contributions/4380251/attachments/2256031/3920589/updated_TopTag_JMAR_06Jul2021.pdf}{\textcolor{kBlue}{JMAR group}}
        %\end{itemize}
        \href{http://cms-results.web.cern.ch/cms-results/public-results/publications/HIG-21-010/index.html}{\textcolor{kBlue}{HIG-21-010}}
        Submitted to JHEP\newline
        \vspace{0.1cm}Documentation: \href{https://cms.cern.ch/iCMS/user/noteinfo?cmsnoteid=CMS\%20AN-2021/019}{\textcolor{kBlue}{AN 2021/019}}\newline
        \vspace{0.1cm}Approved by \href{https://indico.cern.ch/event/1042288/contributions/4380251/attachments/2256031/3920589/updated_TopTag_JMAR_06Jul2021.pdf}{\textcolor{kBlue}{JMAR group}}

      }

      %\PlaceText{73mm}{65mm}{\scriptsize{\textcolor{kBlue}{loose}}}
      %\PlaceText{67mm}{72mm}{\tiny{\textbf{Loose WP}}}
      \PlaceText{67mm}{65mm}{\boxed{\tiny{\textbf{Loose WP}}}}
}
  
\MySubSection{background}
\Slide{t}{

  \vspace{0.1cm}
  
  %\small
  Main background for the \PHpm $\rightarrow$ tb fully hadronic final
  state:

  \vspace{0.1cm}
  
  \begin{itemize}
    \small
  \item QCD multijet \DataDrivenTag
  \item EWK processes (mainly \ttbar) \SimulationTag
  \end{itemize}

  \vspace{0.35cm}

  \normalsize
  \textbf{QCD background measurement}\newline
  \small
  %\textbf{Defining 3 orthogonal control regions (CR) for each SR}:\newline
  %\textbf{Use discriminating variables to define 3 orthogonal control regions (CR) for each SR}:

  \vspace{-0.25cm}
  
  Defining 3 orthogonal control regions (CR) for each SR

  \vspace{-0.2cm}
  \twoColumnsAsymAlt
      {
        \begin{itemize}
        \small
        \item \textcolor{kRed}{$t^{res}_{assoc}$ mass}: In-mass
          $\rightarrow$ Off-mass ``sidebands''
        \item \textcolor{kRed}{$t^{res}_{H^{\pm}}$ mva}: t-tagged (t) $\rightarrow$ non t-tagged (!t)
        \end{itemize}
      }
      {
      }

  \vspace{0.55cm}
      
  ``ABCD'' method
  
  \twoColumnsAsymAlt
      {
        \begin{equation*}
          \boxed{\textcolor{black}{N^{SR}_{QCD}} =
            \sum_{i}^{\mathrm{bins}} \textcolor{black}{N^{CR(off\texttt{-}mass,t)}_{QCD,i}}  \cdot \left(\frac{\textcolor{black}{N^{CR(in\texttt{-}mass,!t)}_{QCD,i}}}{\textcolor{black}{N^{CR(off\texttt{-}mass,!t)}_{QCD,i}}} \right)}
        \end{equation*}

        \small        
        \begin{itemize}
          \footnotesize
        \item Performed in 3 bins of the $t^{res}_{assoc}$ \pT:
          \begin{itemize}
            \footnotesize
          \item \pT $\in$ [0, 100, 300, $\infty$] GeV \FixmeTag
          \end{itemize}          
        \end{itemize}        
      }
      {

        \vspace{-1.6cm}
        
        \oneFigColumns{invertAssocTopDiagram.pdf}{Sidebands}{1.0}
      }

        %\item Shape:  t-tagged $t^{res}_{H^{\pm}}$, off $t^{res}_{assoc}$ mass
        %\item Normal.: non-tagged $t^{res}_{H^{\pm}}$, in/off $t^{res}_{assoc}$ mass      
}

\MySubSection{signal extraction}
\Slide{t}{

  A parameterized DNN is developed to extract signal from SM background
  \begin{itemize}
    \small
  \item \textcolor{blue}{Signal}: $H^{\pm} \rightarrow$ tb for different mass hypotheses
  \item \textcolor{red}{Background}: \ttbar $\rightarrow$ SR,
    Combinatorial $\rightarrow$ CR$^{(off\texttt{-}mass,t)}$ \TTMCTag
    %\begin{itemize}
    %  \footnotesize
    %\item \ttbar from SR
    %\item Combinatorial from CR$^{(off\texttt{-}mass,t)}$ %\textcolor{kGreen}{0 matched-$t_{res}$}
    %\end{itemize}
  %\item Training (test) is done using \ttbar MC of 2017 (2018) data era \OrthogonalityTag
  \end{itemize}

  \twoColumnsAsymOp
  {
    \small
    %\begin{tcolorbox}[colback=white]
    \center{\textbf{\textcolor{oxfordblue}{Input variables}}}
    \vspace{0.1cm}

    \begin{tcolorbox}[colback=white]
    \begin{itemize}
      \CustomiseLineSpacing{0.75}
      \scriptsize
    \item[1] $\Delta \theta$($t_{H+}, b_{H+}$) in $H^{\pm}$ CM
    \item[2] $H_{T,3b}$% = pt($b_{t^{res}_{assoc}}$) + pt($b_{t^{res}_{H^{\pm}}}$) + pt($b_{H^{\pm}}$)
    \item[3] $p_{T}$(bb$_{dRmin}$)
    \item[4] m(bb$_{maxPt}$)
    \item[5] y23 = $p^{2}_{T,j3}/(p_{T,j1}+p_{T,j2})^{2}$%$\frac{p^{2}_{T,j3}}{(p_{T,j1}+p_{T,j2})^{2}}$
    \item[6] $p_{T,b({\PHpm})}$/$H_{T,3b}$%$\frac{p_{T,b_{H+}}}{HTb}$
    \item[7] $m_{H^{\pm}}$
    \item[8] $p_{T}^{Asym}$(\PHpm, b$_{\PHpm}$)
    %\item[8] $\frac{|p_{T,H+} - p_{T,b_{H+}}|}{p_{T,H+} + p_{T,b_{H+}}}$
    \item[9] Circularity
    \item[10] Sphericity
    \item[11] Aplanarity
    \item[12] Number of medium tops
    \item[13] \textcolor{red}{True mass}
      \vspace{-0.15cm}
    \item[]
  \end{itemize}
    \end{tcolorbox}
  }
  {
    \vspace{0.05cm}
    
    \oneFigColumns{parameterizedDNN.png}{\textbf{Parameterized DNN}}{1.0}

    \vspace{0.3cm}

    \begin{itemize}
      \footnotesize
    \item \textcolor{red}{True mass} is the \textcolor{red}{$\theta$} parameter
    \item In background events, the true mass is randomly assigned to the same values used for signal
    \item Training (test) is done using 2017 (2018) data %\OrthogonalityTag
    \end{itemize}
  }    
}

\MySection{Boosted}
\MySubSection{event selection}

\Slide{t}{

  \textbf{Signal region (SR):}  \FixmeTag
  \vspace{-0.2cm}
  %\begin{center}
  %  \resizebox{1.0\linewidth}{!}{
  %    \begin{tabular}{l l}
  %      \hline
  %      \cellcolor[gray]{.9}Trigger  &
  %      \cellcolor[gray]{.9}\begin{tabular}{@{}c@{}} $H_{T}$ +
  %        multijet + 1 or 2 b jets\\  $H_{T}$ + AK8 jet + trim mass
  %        $ \ \ $ \end{tabular}\\
  %      $e$ veto & \pT > 10 GeV, $|\eta|$ < 2.4, Loose miniIso, cutBasedElectronID (veto) \\%\_Fall17\_94X\_V2\_veto \\
  %      \cellcolor[gray]{.9}$\mu$ veto & \cellcolor[gray]{.9}\pT > 10 GeV, $|\eta|$ < 2.4, Loose miniIso isCutBasedIDLoose\\
  %      $\tau$ veto & \pT > 20 GeV, $|\eta|$ < 2.3, DeepTau \dDeepTau[vloose]{\Pe}, \dDeepTau[medium]{\mu}, \dDeepTau[loose]{j}\\
  %      \cellcolor[gray]{.9}$=$ 1 boosted top ($t^{bst}$) &  \cellcolor[gray]{.9}loose ID, \pT > 400 GeV, $|\eta|$ < 2.4, \mSD[] $\in$ [105,210] GeV, \subjettiness[]{32} < 0.67, $N_{b \ \text{subjets}}$ $\ge$ 1\\        
  %      $\ge$ 4 jets & \pT > 40 GeV, $|\eta|$ < 2.4, tight ID, $H_{T}$ > 500 GeV\\
  %      \cellcolor[gray]{.9}$\ge$ 2 b jets & \cellcolor[gray]{.9}\pT > 40 GeV, DeepJet Medium WP \\
  %      $\ge$ 1 $t^{res}$ & custom DNN medium, $130 < m_{\mathrm{t^{res}}} < 210~$GeV\\
  %      \cellcolor[gray]{.9}Kinematic requirements & \cellcolor[gray]{.9}$\Delta R$($t^{bst}$, $b^{ldg}$) > 1.0, $\max$($m_{bb}$) > 200 GeV\\
  %      \hline
  %    \end{tabular}
  %  }
  %\end{center}

  \begin{center}
    \resizebox{1.0\linewidth}{!}{
      \begin{tabular}{l l}
        \hline
        \cellcolor[gray]{.9}Trigger & \cellcolor[gray]{.9}\begin{tabular}{@{}c@{}} $H_{T}$ + multijet + 1 or 2 b jets\\  $H_{T}$ + AK8 jet + trim mass $ \ \ $ \end{tabular}\\
        $\ell$ veto & same as resolved\\
        \cellcolor[gray]{.9}$=$ 1 boosted top ($t^{bst}$) &  \cellcolor[gray]{.9}loose ID, \pT > 400 GeV, $|\eta|$ < 2.4, \mSD[] $\in$ [105,210] GeV, \subjettiness[]{32} < 0.67, $N_{b \ \text{subjets}}$ $\ge$ 1\\        
        $\ge$ 4 jets & \pT > 40 GeV, $|\eta|$ < 2.4, tight ID, $H_{T}$ > 500 GeV\\
        \cellcolor[gray]{.9}$\ge$ 2 b jets & \cellcolor[gray]{.9}\pT > 40 GeV, DeepJet Medium WP \\
        $\ge$ 1 $t^{res}$ & same as resolved\\
        \cellcolor[gray]{.9}Kinematic requirements & \cellcolor[gray]{.9}\begin{tabular}{@{}c@{}}$\Delta R$($t^{bst}$, $b^{ldg}$) > 1.0\\ $\max$($m_{bb}$) > 200 GeV \end{tabular}\\
        \hline
      \end{tabular}
    }
  \end{center}
  
  \vspace{0.75cm}
  
  
  \twoColumnsAsymAlt
      {
        \normalsize
        $\ \ $ \textbf{SR categorization} \FixmeTag
        \begin{itemize}
          \small
        \item 
        \end{itemize}

        \vspace{0.3cm}

        $\ \ $ \textbf{Invariant \PHpm~mass reconstruction}:
        \begin{itemize}
          \large
        \item[] \textcolor{kRed}{\mHpmReco = $t^{bst}$ + bjet$^{ldg \ p_{T}}$}
        \end{itemize}
      }
      {
        \vspace{-0.4cm}
        \oneFigColumns{tikz/pdf/HPlus4FS_HToTB_BoostedTop_InvMass}{}{1.0}%\mHpmReco=\rTop$_{ \ ldg \ p_{T}}$ + bjet$^{free}_{ldg \ p_{T}}$}{0.68}
      }
}
