%%%%%%%%%%%%%%%%%%%%%%%%%%%%%%%%%%%%%%%%%%%%%%%%%%%%%%%%%%%%%%%%%%%%%%%%%%%%%%%%%%%%%%%%%%%%%%%%%%%%%%%%%
%% File .......... : BeamerCommands.tex
%% Author ........ : Alexandros X. Attikis
%% Institute ..... : University of Cyprus (UCY)
%% e-Mail ........ : attikis@cern.ch
%%%%%%%%%%%%%%%%%%%%%%%%%%%%%%%%%%%%%%%%%%%%%%%%%%%%%%%%%%%%%%%%%%%%%%%%%%%%%%%%%%%%%%%%%%%%%%%%%%%%%%%%%
\newcommand{\en}{\selectlanguage{english}}
\newcommand{\gr}{\selectlanguage{greek}}
        
%%%%%%%%%%%%%%%%%%%%%%%%%%%%%%%%%%%%%%%%%%%%%%%%%%%%%%%%%%%%%%%%%%%%%%%%%%%%%%%%%%%%%%%%%%%%%%%%%%%%%%%%%
% Colour definition (http://en.wikipedia.org/wiki/Web_colors)
%%%%%%%%%%%%%%%%%%%%%%%%%%%%%%%%%%%%%%%%%%%%%%%%%%%%%%%%%%%%%%%%%%%%%%%%%%%%%%%%%%%%%%%%%%%%%%%%%%%%%%%%%
\definecolor{kWhite}         {RGB}{255, 255, 255}
\definecolor{kSnow}          {RGB}{250, 250, 250}
\definecolor{kBlack}         {RGB}{  0,   0,   0}
\definecolor{kPink}          {RGB}{227,  74, 147}
\definecolor{kGray}          {RGB}{192, 192, 192}
\definecolor{kNavyBlue}      {RGB}{28 , 130, 185}
\definecolor{kDarkBlue}      {RGB}{  0,   0, 214}
\definecolor{kLightBlue}     {RGB}{  0,  61, 245}
\definecolor{kvLightBlue}    {RGB}{  0, 184, 245}
\definecolor{kMyBlue}        {RGB}{ 51, 102, 255}
\definecolor{kBlue}          {RGB}{ 51, 102, 255}
\definecolor{kCyan}          {RGB}{ 153, 255, 214}

\definecolor{kGreen}         {RGB}{ 98, 158,  31}
\definecolor{kDarkGreen}     {RGB}{ 50, 133,   0}
\definecolor{kRed}           {RGB}{220,   0,   0}
\definecolor{kOrange}        {RGB}{230, 120,  20}
\definecolor{kYellow}        {RGB}{255, 221,   0}
\definecolor{kBlue}          {RGB}{ 10,  50, 150}
\definecolor{kBrown}         {RGB}{120,  89,  30}
\definecolor{kPink}          {RGB}{255,  0 , 128}
\definecolor{kPurple}        {RGB}{148,  0 , 211}
\definecolor{kMagenta}       {RGB}{255,  0 , 255}
\definecolor{kHLTausBlue}    {RGB}{ 11,  36, 251} % 0B24FB
\definecolor{kHLTausGreen}   {RGB}{223, 254, 191} % DFFEBF
\definecolor{kHLTausOrange}  {RGB}{255, 166,  77} % FFA64D
\definecolor{kHLTausRed}     {RGB}{255,  51,  34} % FF3322
\definecolor{kHPlusBlue}     {RGB}{ 11,  36, 251}
\definecolor{kHPlusLightBlue}{RGB}{136, 210, 249}
\definecolor{kHPlusYellow}   {RGB}{254, 222, 134}
\definecolor{kHPlusBlack}    {RGB}{  6,   6,   7}
\definecolor{kHPlusWhite}    {RGB}{255, 255, 255}
\definecolor{kHPlusDarkBlue} {RGB}{ 15,  54, 148}
%
\definecolor{kPSModules}     {RGB}{  7,  71, 132}
\definecolor{k2SModules}     {RGB}{252,  68,  35}
\definecolor{kPixels}        {RGB}{ 74,  35, 109}
\definecolor{kTRACKER}{RGB}{124, 181, 204}
\definecolor{kECAL}{RGB}{98, 189, 156}
\definecolor{kHCAL}{RGB}{244, 239, 146}
\definecolor{kMAGNET}{RGB}{255, 255, 255}
\definecolor{kYOKE}{RGB}{176, 45, 44}
\definecolor{kMUON}{RGB}{190, 190, 190}
\definecolor{kBASELINE}{RGB}{ 11,  36, 251}
\definecolor{kINVERTED}{RGB}{255,  51,  34}
\definecolor{QCD}{RGB}{254, 203, 69}
\definecolor{TT}{RGB}{11, 35, 243}
\definecolor{DYJetsToQQHT}{RGB}{157, 254, 205}
\definecolor{TTWJetsToQQ}{RGB}{154, 202, 39}
\definecolor{WJetsToQQHT}{RGB}{202, 52, 21}
\definecolor{SingleTop}{RGB}{31, 202, 62}
\definecolor{Diboson}{RGB}{54, 63, 251}
\definecolor{TTZToQQ}{RGB}{105, 155, 252}
\definecolor{TTTT}{RGB}{255, 254, 159}
\definecolor{kPostitYellow}{RGB}{242,226,149}

%==============================================
% Soti's colors
%==============================================

\definecolor{indianred}{rgb}{0.85, 0.4, 0.4} %0.8, 0.36, 0.36
\definecolor{lightsalmonpink}{rgb}{1.0, 0.6, 0.6}
\definecolor{apricot}{rgb}{0.98, 0.81, 0.69}
\definecolor{floralwhite}{rgb}{1.35, 0.9, 0.9} %1.0, 0.98, 0.94
\definecolor{darkgray}{rgb}{0.43, 0.5, 0.5}
\definecolor{oxfordblue}{rgb}{0.0, 0.13, 0.28}
\definecolor{wildblueyonder}{rgb}{0.64, 0.68, 0.82}
\definecolor{deepjunglegreen}{rgb}{0.0, 0.29, 0.29}
\definecolor{aliceblue}{rgb}{0.94, 0.97, 1.0}
\definecolor{beaublue}{rgb}{0.74, 0.83, 0.9}


\newcommand{\TRACKER}[1]{\TextColour{kTRACKER}{#1}}
\newcommand{\ECAL}[1]{\TextColour{kECAL}{#1}}
\newcommand{\HCAL}[1]{\TextColour{kHCAL}{#1}}
\newcommand{\MAGNET}[1]{\TextColour{kMAGNET}{#1}}
\newcommand{\YOKE}[1]{\TextColour{kYOKE}{#1}}
\newcommand{\MUON}[1]{\TextColour{kMUON}{#1}}
\newcommand{\BASELINE}[1]{\TextColour{kBASELINE}{#1}}
\newcommand{\INVERTED}[1]{\TextColour{kINVERTED}{#1}}
\newcommand{\cTT}[1]{\TextColour{TT}{#1}}
\newcommand{\cDYJetsToQQHT}[1]{\TextColour{DYJetsToQQHT}{#1}}
\newcommand{\cTTWJetsToQQ}[1]{\TextColour{TTWJetsToQQ}{#1}}
\newcommand{\cWJetsToQQHT}[1]{\TextColour{WJetsToQQHT}{#1}}
\newcommand{\cSingleTop}[1]{\TextColour{SingleTop}{#1}}
\newcommand{\cDiboson}[1]{\TextColour{Diboson}{#1}}
\newcommand{\cTTZToQQ}[1]{\TextColour{TTZToQQ}{#1}}
\newcommand{\cTTTT}[1]{\TextColour{TTTT}{#1}}

%%%%%%%%%%%%%%%%%%%%%%%%%%%%%%%%%%%%%%%%%%%%%%%%%%%%%%%%%%%%%%%%%%%%%%%%%%%%%%%%%%%%%%%%%%%%%%%%%%%%%%%%%
% Analysis-specific commands
%%%%%%%%%%%%%%%%%%%%%%%%%%%%%%%%%%%%%%%%%%%%%%%%%%%%%%%%%%%%%%%%%%%%%%%%%%%%%%%%%%%%%%%%%%%%%%%%%%%%%%%%%
\newcommand{\hCMSSW}[1]{\MyHRef{http://cmsdoxygen.web.cern.ch/cmsdoxygen/}{CMSSW\_#1}}
\newcommand{\PPCollisionsThirteenTeV}{\MyHRef{https://twiki.cern.ch/twiki/bin/view/CMSPublic/LumiPublicResults\#2016\_Proton\_Proton\_13\_TeV\_Collis}{2016 Proton-Proton 13 \TeV Collisions}\xspace}
\newcommand{\ReMiniAOD}{\MyHRef{https://twiki.cern.ch/twiki/bin/view/CMSPublic/WorkBookMiniAOD\#2016\_Data\_for\_Moriond\_2017}{}\xspace}
\newcommand{\FebReMiniAOD}{\MyHRef{https://twiki.cern.ch/twiki/bin/view/CMSPublic/WorkBookMiniAOD\#2016_Data_for_Moriond_2017}{03Feb2017 Re-miniAOD}\xspace}
\newcommand{\AprilReReco}{\MyHRef{https://hypernews.cern.ch/HyperNews/CMS/get/offline-dqm/671/1/1/1/1.html}{18Apr2017 ReReco}\xspace}
\newcommand{\AugReReco}{\MyHRef{https://hypernews.cern.ch/HyperNews/CMS/get/dataopsrequests/19907.html}{07Aug2017 Legacy ReReco}\xspace}
\newcommand{\MoriondTwentySeventeen}{\MyHRef{https://twiki.cern.ch/twiki/bin/view/CMSPublic/WorkBookMiniAOD\#Run2\_Moriond17\_re\_digi\_reco\_camp}{Moriond17 MC}\xspace}
\newcommand{\SiStripHitEffLoss}{\MyHRef{https://twiki.cern.ch/twiki/bin/view/CMS/SiStripHitEffLoss}{SiStripHitEffLoss}}
\newcommand{\DASQueryForJetHT}{\MyHRef{https://cmsweb.cern.ch/das/request?view=list\&limit=50\&instance=prod\%2Fglobal\&input=dataset\%3D\%2FJetHT\%2FRun2016*-03Feb2017*-v*\%2FMINIAOD}{dataset=/JetHT/Run2016*-03Feb2017*-v*/MINIAOD}}

%%%%%%%%%%%%%%%%%%%%%%%%%%%%%%%%%%%%%%%%%%%%%%%%%%%%%%%%%%%%%%%%%%%%%%%%%%%%%%%%%%%%%%%%%%%%%%%%%%%%%%%%%
% How to place images behind text !!! (z-order)
%%%%%%%%%%%%%%%%%%%%%%%%%%%%%%%%%%%%%%%%%%%%%%%%%%%%%%%%%%%%%%%%%%%%%%%%%%%%%%%%%%%%%%%%%%%%%%%%%%%%%%%%%
% (http://tex.stackexchange.com/a/134311)
\makeatletter
\newbox\@backgroundblock
\newenvironment{backgroundblock}[2]{%
  \global\setbox\@backgroundblock=\vbox\bgroup%
    \unvbox\@backgroundblock%
    \vbox to0pt\bgroup\vskip#2\hbox to0pt\bgroup\hskip#1\relax%
}{\egroup\egroup\egroup}
\addtobeamertemplate{background}{\box\@backgroundblock}{}
\makeatother

%%%%%%%%%%%%%%%%%%%%%%%%%%%%%%%%%%%%%%%%%%%%%%%%%%%%%%%%%%%%%%%%%%%%%%%%%%%%%%%%%%%%%%%%%%%%%%%%%%%%%%%%%
% Customise tcolorboxes
%%%%%%%%%%%%%%%%%%%%%%%%%%%%%%%%%%%%%%%%%%%%%%%%%%%%%%%%%%%%%%%%%%%%%%%%%%%%%%%%%%%%%%%%%%%%%%%%%%%%%%%%%
\tcbset{
    noparskip,
    coltitle  = green,
    colframe  = pink!10,
    colback   = orange!30,
    coltext   = black,
    fonttitle = \footnotesize,
    figureStyle/.style      = { coltitle=\MyMainTextColour, colframe=white, colback=white, coltext=white},
    transparentStyle/.style = { standard jigsaw, opacityframe=0.0, opacityback=0.0},
    alertStyle/.style       = { coltitle=\MyMainTextColour, colframe=gray!40, colback=gray!40},
    animationStyle/.style   = { coltitle=\MyMainTextColour, colframe=white, colback=white, coltext=white},
    boxsep      = -0.00cm,
    top         = +0.00cm,
    bottom      = +0.00cm,
    left        = +0.00cm,
    right       = +0.00cm,
    toptitle    = +0.00cm,
    bottomtitle = +0.00cm,
    nobeforeafter,
    center title,
    arc=0cm, % 0cm for non-rounded corners!
}
    
%%%%%%%%%%%%%%%%%%%%%%%%%%%%%%%%%%%%%%%%%%%%%%%%%%%%%%%%%%%%%%%%%%%%%%%%%%%%%%%%%%%%%%%%%%%%%%%%%%%%%%%%%
% Theme Customisations (http://en.wikibooks.org/wiki/LaTeX/Presentations)
%%%%%%%%%%%%%%%%%%%%%%%%%%%%%%%%%%%%%%%%%%%%%%%%%%%%%%%%%%%%%%%%%%%%%%%%%%%%%%%%%%%%%%%%%%%%%%%%%%%%%%%%%
\newcommand{\CustomiseColours}[6]{
  \newcommand{\MyMainColour}{#1}
  \newcommand{\MySectionColour}{#2}
  \newcommand{\MySubsectionColour}{#3}
  \newcommand{\MyMainTextColour}{#4}
  \newcommand{\MySectionTextColour}{#5}
  \newcommand{\MySubsectionTextColour}{#6}
}

\newcommand{\ColourA}{kHLTausRed}
\newcommand{\ColourB}{kBlue} %kHLTausOrange}
\newcommand{\ColourC}{kDarkGreen}
\newcommand{\ColourD}{kHLTausOrange}

\newcommand{\alertA}[1]{\TextColour{\ColourA}{#1}}
\newcommand{\alertB}[1]{\TextColour{\ColourB}{#1}}
\newcommand{\alertC}[1]{\TextColour{\ColourC}{#1}}
\newcommand{\alertD}[1]{\TextColour{\ColourD}{#1}}

%%% Title & Section colours
\setbeamercolor{title}{fg=\MyMainTextColour, bg=\MyMainColour}
\setbeamercolor{background canvas}{bg=white}
\setbeamercolor{normal text}{fg=\MyMainTextColour}
\setbeamercolor{section in head/foot}{fg=\MySectionTextColour, bg=\MySectionColour}
\setbeamercolor{subsection in head/foot}{fg=\MySubsectionTextColour, bg=\MySubsectionColour}

\newcommand\textbox[1]{\parbox{.32\textwidth}{#1}}

%%% Customise text margins
\newcommand{\CustomiseTextMargins}[2]{%
  \setbeamersize{text margin left=#1}
  \setbeamersize{text margin right=#2}
}

%%% Customise tables: width of rule
\newcommand{\CustomiseTables}[1]{ \setlength{\arrayrulewidth}{#1}}

%%% Customise text format. 10/5/1999 or 10.5.1999 or what?
\newcommand{\CustomiseDateFormat}[1]{\renewcommand{\dateseparator}{#1}}

%%% Customise line spacing (global)
\newcommand{\CustomiseLineSpacing}[1]{\setstretch{#1}}

%%% Make title page
\newcommand{\MakeTitlePage}[5]{%
  \title{\MyTitle}
  \subtitle{\MySubtitle}
  \author{\MyAuthors}
  \institute{\MyInstitutes}
  \date{\MyDate}
  \maketitle
}
  
%%% Customise Talk Theme (https://en.wikibooks.org/wiki/LaTeX/Presentations)
\newcommand{\CustomiseTheme}[7]{
  \setbeamertemplate{blocks}[rounded][shadow=false] 
  \setbeamercolor{uppercol}{fg=white, bg=\MySectionColour}
  \setbeamercolor{lowercol}{fg=red, bg=orange}
  \setbeamercolor{block title}{fg=white, bg=kGreen}	
  \setbeamercolor{block body}{fg=black, bg=kGreen!20}
  \setbeamercolor{block title alerted}{fg=white, bg=kOrange}
  \setbeamercolor{block body alerted}{fg=black, bg=kOrange!20}
  \setbeamercolor{itemize item}{fg=\MySectionColour} % bullet colour  
  %%% Navigation Bar
  \ifbool{Navigation}{%
    \setbeamercolor{navigation symbols}{fg=\MySubsectionColour, bg=\MySubsectionColour}%
  }{% Remove navigation symbols
    \setbeamertemplate{navigation symbols}{}
  }
  \mode<presentation>  
  %%% Theme options
  \usetheme[#1]{#2} % #1= Options, #2 = Theme (Dresden, Madrid, Frankfurt, ...) Not all options are supported by all themes
  \useoutertheme{split}
  \usecolortheme[named=\MySectionColour]{structure}
  \useinnertheme{rounded}             % Usecase: customise enumerate numbering icon (rectangles, circles, inmargin, rounded)
  \setbeamercovered{transparent}      % Usecase: enables transparent overlays
  %%% Colour palettes
  \setbeamercolor{palette primary}{fg=red, bg=orange}
  \setbeamercolor{palette secondary}{fg=white, bg=\MySectionColour}
  \setbeamercolor{palette tertiary}{fg=\MySectionTextColour, bg=\MyMainColour}
  \setbeamercolor{palette quaternary}{fg=red, bg=orange}
  \setbeamercolor{button}{bg=\MySectionColour,fg=white}
  \setbeamertemplate{itemize items}[#3]  % Examples: default, triangle, circle, square, ball, $\int$, \tikz\node[cross out, draw] {good} 
  \newcommand{\LeftSkip}{0.2cm}          % Default: 0.2cm
  \newcommand{\RightSkip}{0.2cm}         % Default: 0.2cm
  %%% Header/Footer
  \ifbool{SubsectionBullets}{%
    \setbeamertemplate{headline}{%
      \begin{beamercolorbox}[ht=2.5ex,dp=3.75ex]{section in head/foot}
        \insertnavigation{\paperwidth}
      \end{beamercolorbox}%
      % Comment 3 lines below to disable sub-section title bar in header
      \begin{beamercolorbox}[ht=2.2ex,dp=1.0ex, leftskip=.3cm,rightskip=.3cm plus1fil]{subsection in head/foot}
        \usebeamerfont{subsection in head/foot} \insertsubsectionhead
      \end{beamercolorbox}%
    }%
  }{% See https://tex.stackexchange.com/questions/243954/modify-main-slide-dimensions-beamer
    \setbeamertemplate{headline}{%
      \setbeamerfont*{section in head/foot}{size=\normalsize}
      \begin{beamercolorbox}[ht=\HeaderHeight, wd=1.0\paperwidth, dp=\FooterBoxDepth,leftskip=\LeftSkip, rightskip=\RightSkip]{section in head/foot}
        \usebeamerfont{section in head/foot}
        %\bf\MakeUppercase{\insertsectionhead} \it
        %\textcolor{\MySubsectionColour}{\insertsubsectionhead}
        \bf\MakeUppercase{\insertsectionhead} \textcolor{\MySubsectionColour}{\insertsubsectionhead}
      \end{beamercolorbox}%
    }
  }
  %%% Footer
  \ifbool{Footer}{
    \newcommand{\HeaderHeight}{0.45cm}  % default: 0.4cm
    \newcommand{\FooterHeight}{0.20cm}  % default: 0.2cm
    \newcommand{\FooterBoxDepth}{0.1cm} % default: 0.1cm (the distance between the baseline and the bottom of the box)
    \setbeamertemplate{footline}{%
      \leavevmode
      \hbox{%
        \fontsize{11}{13}\selectfont%
        \begin{beamercolorbox}[wd=1.0\paperwidth, ht=\FooterHeight, dp=\FooterBoxDepth, leftskip=\LeftSkip, rightskip=\RightSkip]{title in head/foot}
          \scriptsize \noindent\textbox{#4\hfill}\textbox{\hfil\insertframenumber\ of \inserttotalframenumber\hfil}\textbox{\hfill #5}
          % Place navigation bar in footer?
          % \footnotesize \noindent\textbox{#1\hfill}\textbox{\hfil\insertframenumber\ of \inserttotalframenumber\hfil}\textbox{\hfill 
          %   \insertslidenavigationsymbol
          %   \insertframenavigationsymbol%
          %   \insertsectionnavigationsymbol%
          %   \insertsubsectionnavigationsymbol%
          %   \insertsectionnavigationsymbol%
          %   \insertdocnavigationsymbol%
          %   \insertbackfindforwardnavigationsymbol%
        \end{beamercolorbox}%   
      }
    }
  }{%
    \setbeamertemplate{footline}{}
  }
  %%% Title page
  % \usefonttheme{structurebold} #no idea what this does
  \setbeamerfont{title}{size=\large,series=\bfseries,parent=structure}
  \setbeamerfont{subtitle}{size=\normalsize,series=\normalfont,parent=structure}
  \setbeamerfont{author}{size=\normalsize,parent=structure}
  \setbeamerfont{institute}{size=\scriptsize,parent=structure}
  \setbeamerfont{date}{size=\normalsize,parent=structure}
  \setbeamertemplate{title page}[default][rounded=false,center,colsep=0.2cm]
  %%% Notes (Use notes in  slides with: \note{this is a note}
  \ifbool{Notes}{\setbeameroption{show notes on second screen}}{}
  %%% Set a gradient background colour for all slides?
  \ifbool{GradientBackground}{\beamertemplateshadingbackground{#6}{#7}}{}
}

%%%%%%%%%%%%%%%%%%%%%%%%%%%%%%%%%%%%%%%%%%%%%%%%%%%%%%%%%%%%%%%%%%%%%%%%%%%%%%%%%%%%%%%%%%%%%%%%%%%%%%%%%
% Overlays
%%%%%%%%%%%%%%%%%%%%%%%%%%%%%%%%%%%%%%%%%%%%%%%%%%%%%%%%%%%%%%%%%%%%%%%%%%%%%%%%%%%%%%%%%%%%%%%%%%%%%%%%%
  \newcommand{\OverlayFrame}[3]{
  \only<#2>{
    \begin{tikzpicture}[remember picture,overlay]%
      \draw[fill= #3] % e.g. black, opacity=0.8 
      (current page.north east) rectangle (current page.south west);
      \node at (current page.center) {
          \begin{minipage}{1.0\linewidth}  
	    #1	
          \end{minipage}
      };
    \end{tikzpicture}
    \normalsize
  }
}
  
\newcommand{\MessageFrame}[3]{
  \only<#1>{
    \begin{tikzpicture}[remember picture,overlay]%
      \draw[fill=black,opacity=0.80] 
      (current page.north east) rectangle (current page.south west);
      \node at (current page.center) {
        %%% What to do
          \begin{minipage}{0.8\linewidth}  
            \begin{block}{\Huge\centering\SpecialColour{#2}}
              \Large #3
            \end{block}
          \end{minipage}
      };
    \end{tikzpicture}
    \normalsize
  }
}

\newcommand{\OverlayRectangle}[7]{
  #1{
    \begin{tikzpicture}[remember picture,overlay]
      \draw[fill=#3,opacity=#2](#4,#5) rectangle (#6,#7);
    \end{tikzpicture}
    \normalsize
  }
}

%%% Reveal the TikZ grid of the slide
\newcommand{\RevealTikzGrid}{
    \begin{tikzpicture}

      \foreach \i [evaluate=\i] in {0,1,2,...,11} {
        \draw(0,0) rectangle (\i,\i);
        \foreach \j [evaluate=\i] in {0,1,2,...,12} {
          \draw(\j,\j) rectangle (\i,\i);
        }
      }
    \end{tikzpicture}
}


\newcommand{\RectangleText}[1]{
  \begin{tikzpicture}
    \node[align=center,kGreen!50, draw,fill, text=white, rectangle, cloud puffs=12, aspect=3.0, scale=0.5] at (0,0) {\textbf{\Large \bf{#1}}};
\end{tikzpicture}
}

%%%%%%%%%%%%%%%%%%%%%%%%%%%%%%%%%%%%%%%%%%%%%%%%%%%%%%%%%%%%%%%%%%%%%%%%%%%%%%%%%%%%%%%%%%%%%%%%%%%%%%%%%
% Arrows
%%%%%%%%%%%%%%%%%%%%%%%%%%%%%%%%%%%%%%%%%%%%%%%%%%%%%%%%%%%%%%%%%%%%%%%%%%%%%%%%%%%%%%%%%%%%%%%%%%%%%%%%%
\tikzset{
    myarrow/.style={
        draw,
        fill=blue,
        single arrow,
        minimum height=3.5ex,
        single arrow head extend=1ex
    }
}

\newcommand{\arrowleft}{%
\tikz [baseline=-0.5ex]{\node [myarrow,rotate=0] {};}
}
\newcommand{\arrowright}{%
\tikz [baseline=-0.5ex]{\node [myarrow,rotate=180] {};}
}
\newcommand{\arrowup}{%
\tikz [baseline=-0.5ex]{\node [myarrow,rotate=90] {};}
}
\newcommand{\arrowdown}{%
\tikz [baseline=-1ex]{\node [myarrow,rotate=-90] {};}
}

\newcommand\tikzmark[1]{%
  \tikz[remember picture,overlay]\node (#1) {};%
}

\newcommand\Connect[3][]{%
\tikz[remember picture,overlay]
  \draw[->,red,>=latex,#1] (#2.north east) -- ( $ (#3.north west) + (-20pt,0) $ );%
}


%%%%%%%%%%%%%%%%%%%%%%%%%%%%%%%%%%%%%%%%%%%%%%%%%%%%%%%%%%%%%%%%%%%%%%%%%%%%%%%%%%%%%%%%%%%%%%%%%%%%%%%%%
% Figures
%%%%%%%%%%%%%%%%%%%%%%%%%%%%%%%%%%%%%%%%%%%%%%%%%%%%%%%%%%%%%%%%%%%%%%%%%%%%%%%%%%%%%%%%%%%%%%%%%%%%%%%%%
%%% 1x1 figure (full page)
\newcommand{\oneFigMax}[1]{
  \clearpage
  \begin{tikzpicture}[remember picture,overlay]
    \node[at=(current page.center),yshift=-0.5cm] {
      \includegraphics[width=1.0\paperwidth,keepaspectratio]{./figures/#1}
    };
  \end{tikzpicture}
}

%%% 1x1 figure (with label)
\newcommand{\oneFigLabel}[3]{
  \begin{figure}[!ht]
    \begin{center}
      \includegraphics[width=#3\textwidth,keepaspectratio]{./figures/#2}
      \label{fig:#1}
    \end{center}
  \end{figure}
}


%%% 1x1 figure
\newcommand{\oneFig}[2]{
  \begin{figure}[!ht]
    \begin{center}
      \includegraphics[width=#2\textwidth,keepaspectratio]{./figures/#1}
    \end{center}
  \end{figure}
}


%%% 1x1 figure (no label)
\newcommand{\oneFigNoLabel}[2]{
  \begin{figure}[!ht]
    \begin{center}
      \includegraphics[width=#2\textwidth,keepaspectratio]{./figures/#1}
    \end{center}
  \end{figure}
}

%%% 2x1 figure
\newcommand{\twoFig}[4]{
\begin{columns}[T] %For stable vertical alignment of columns
  \begin{column}{0.5\textwidth}
    \oneFig{#1}{#2}
  \end{column}
  \begin{column}{0.5\textwidth}
    \oneFig{#3}{#4}
  \end{column}
\end{columns}
}


%%% 1x3 figures
\newcommand{\threeFig}[4]{
  \begin{columns}[T] %For stable vertical alignment of columns
    \begin{column}{0.32\textwidth}
      \oneFig{#1}{#4}
    \end{column}
    \begin{column}{0.32\textwidth}
      \oneFig{#2}{#4}
    \end{column}
    \begin{column}{0.32\textwidth}
      \oneFig{#3}{#4}
    \end{column}
  \end{columns}
}


%%% 2x2 figures
\newcommand{\fourFig}[9]{
\begin{columns}[T] %For stable vertical alignment of columns
  \begin{column}{0.5\textwidth}
    \oneFigLabel{#1_a}{#2}{#3}
  \end{column}
  \begin{column}{0.5\textwidth}
    \oneFigLabel{#1_b}{#4}{#5}
  \end{column}
\end{columns}
\begin{columns}[T] %For stable vertical alignment of columns
  \begin{column}{0.5\textwidth}
    \oneFigLabel{#1_c}{#6}{#7}
  \end{column}
  \begin{column}{0.5\textwidth}
    \oneFigLabel{#1_d}{#8}{#9}
  \end{column}
\end{columns}
}


%%% 3x3 figures
\newcommand{\sixFig}[7]{
    \begin{columns}[T] %For stable vertical alignment of columns
    \begin{column}{0.295\textwidth}
     \oneFigLabel{#1_a}{#2}{1.0} 
    \end{column}
    \begin{column}{0.295\textwidth}
	\oneFigLabel{#1_b}{#3}{1.0} 
    \end{column}
    \begin{column}{0.295\textwidth}
       \oneFigLabel{#1_c}{#4}{1.0}
    \end{column}
  \end{columns}
    \begin{columns}[T] %For stable vertical alignment of columns
    \begin{column}{0.295\textwidth}
	\oneFigLabel{#1_d}{#5}{1.0} 
    \end{column}
    \begin{column}{0.295\textwidth}
 	\oneFigLabel{#1_e}{#6}{1.0} 
    \end{column}
    \begin{column}{0.295\textwidth}
	\oneFigLabel{#1_f}{#7}{1.0} 
    \end{column}
  \end{columns}
}

%%%%%%%%%%%%%%%%%%%%%%%%%%%%%%%%%%%%%%%%%%%%%%%%%%%%%%%%%%%%%%%%%%%%%%%%%%%%%%%%%%%%%%%%%%%%%%%%%%%%%%%%%
% Figures inside tcolorboxes
%%%%%%%%%%%%%%%%%%%%%%%%%%%%%%%%%%%%%%%%%%%%%%%%%%%%%%%%%%%%%%%%%%%%%%%%%%%%%%%%%%%%%%%%%%%%%%%%%%%%%%%%%
%%% Need phantom character to control height of title box 
%%% (different text can have different height and causes vertical-alignment problems)
\newcommand{\tColorBoxTitle}[1]{\SpecialColour{#1\vphantom{\Large$\left(\right)$}}}

%%% rows x columns = 1 x 1 figures (1 titles)
\newcommand{\oneFigColumns}[3]{
  \begin{columns}[T] %For stable vertical alignment of columns
      \begin{column}{#3\textwidth}
        \begin{tcolorbox}[transparentStyle, title=\tColorBoxTitle{#2}]
          \oneFig{#1}{1.0}
        \end{tcolorbox}
      \end{column}
  \end{columns}
}

%%% rows x columns = 1 x 1 figures (1 titles)
\newcommand{\oneFigColumnsAndText}[4]{
  \begin{columns}[T] %For stable vertical alignment of columns
      \begin{column}{0.5\textwidth}
        {#4}
      \end{column}
      \begin{column}{0.5\textwidth}
        \begin{tcolorbox}[transparentStyle, title=\tColorBoxTitle{#2}]
          \oneFig{#1}{#3}
        \end{tcolorbox}
      \end{column}
  \end{columns}
}

%%% rows x columns = 1 x 1 figures (1 titles)
\newcommand{\oneFigColumnsAndTextAlt}[4]{
  \begin{columns}[T] %For stable vertical alignment of columns
      \begin{column}{0.5\textwidth}
        \begin{tcolorbox}[transparentStyle, title=\tColorBoxTitle{#2}]
          \oneFig{#1}{#3}
        \end{tcolorbox}
      \end{column}
      \begin{column}{0.5\textwidth}
        {#4}
      \end{column}
  \end{columns}
}

%%% rows x columns = 1 x 2 figures (2 titles)
\newcommand{\twoFigColumns}[5]{
  \begin{columns}[T] %For stable vertical alignment of columns
      \begin{column}{0.5\textwidth}
        %\begin{tcolorbox}[transparentStyle, title=\tColorBoxTitle{#2}]
        \begin{tcolorbox}[transparentStyle, title=\tColorBoxTitle{#2}]
          \oneFig{#1}{#5}
    \end{tcolorbox}
      \end{column}
      \begin{column}{0.5\textwidth}
        %\begin{tcolorbox}[transparentStyle, title=\tColorBoxTitle{#4}]
        \begin{tcolorbox}[transparentStyle, title=\tColorBoxTitle{#4}]
        \oneFig{#3}{#5}
    \end{tcolorbox}       
      \end{column}
  \end{columns}
}

%%% rows x columns = 2 x 1 figures (2 titles)
\newcommand{\twoFigColumnsVertical}[5]{
  \begin{columns}[T] %For stable vertical alignment of columns
      \begin{column}{1.0\textwidth}
        \begin{tcolorbox}[transparentStyle, title=\tColorBoxTitle{#2}]
          \oneFig{#1}{#5}
        \end{tcolorbox}
        \vspace{0.5cm}
        \begin{tcolorbox}[transparentStyle, title=\tColorBoxTitle{#4}]
          \oneFig{#3}{#5}
        \end{tcolorbox}
        \end{column}
  \end{columns}
}

%%% rows x columns = 2 x 1 figures (2 titles)
\newcommand{\twoFigVertical}[4]{
  \begin{columns}[T] %For stable vertical alignment of columns
      \begin{column}{1.0\textwidth}
        \begin{tcolorbox}[transparentStyle]
          \oneFig{#1}{#2}
        \end{tcolorbox}
          \vspace{0.5cm}
          \begin{tcolorbox}[transparentStyle]
            \oneFig{#3}{#4}
            \end{tcolorbox}
        \end{column}
  \end{columns}
}

%%% rows x columns = 1 x 2 figures (2 titles)
\newcommand{\twoFigColumnsArrowLeft}[5]{
  \begin{columns}[T] %For stable vertical alignment of columns
      \begin{column}{0.45\textwidth}
        \begin{tcolorbox}[transparentStyle, title=\tColorBoxTitle{#2}]
          \oneFig{#1}{#5}
    \end{tcolorbox}
      \end{column}
      \begin{column}{0.1\textwidth}
        \begin{minipage}[c][0.6\textheight][c]{\linewidth}
          \huge\arrowleft
        \end{minipage}
      \end{column}
      \begin{column}{0.45\textwidth}
        \begin{tcolorbox}[transparentStyle, title=\tColorBoxTitle{#4}]
          \oneFig{#3}{#5}
    \end{tcolorbox}       
      \end{column}
  \end{columns}
}

%%% rows x columns = 1 x 2 figures (2 titles)
\newcommand{\twoFigColumnsSize}[6]{
  \begin{columns}[T] %For stable vertical alignment of columns
      \begin{column}{0.5\textwidth}
        \begin{tcolorbox}[transparentStyle, title=\tColorBoxTitle{#2}]
          \oneFig{#1}{#3}
    \end{tcolorbox}
      \end{column}
      \begin{column}{0.5\textwidth}
        \begin{tcolorbox}[transparentStyle, title=\tColorBoxTitle{#5}]
          \oneFig{#4}{#6}
    \end{tcolorbox}       
      \end{column}
  \end{columns}
}

%%% rows x columns = 1 x 3 figures (3 titles)
\newcommand{\threeFigColumns}[6]{    %soti changed size - temporary
  \begin{columns}[T] %For stable vertical alignment of columns
      \begin{column}{0.32\textwidth}
        \begin{tcolorbox}[transparentStyle, title=\tColorBoxTitle{#2}]
          \oneFig{#1}{0.74}
        \end{tcolorbox}
      \end{column}
      %\begin{column}{0.05\textwidth}\end{column}
      \begin{column}{0.32\textwidth}
        \begin{tcolorbox}[transparentStyle, title=\tColorBoxTitle{#4}]
          \oneFig{#3}{0.74}
        \end{tcolorbox}
      \end{column}
      %\begin{column}{0.05\textwidth}\end{column}
      \begin{column}{0.32\textwidth}
        \begin{tcolorbox}[transparentStyle, title=\tColorBoxTitle{#6}]
          \oneFig{#5}{0.74}
        \end{tcolorbox}       
      \end{column}
  \end{columns}
}


%%soti added this temporary
%%% rows x columns = 1 x 3 figures (3 titles)
\newcommand{\threeFigColumnsBig}[6]{    %soti changed size - temporary
  \begin{columns}[T] %For stable vertical alignment of columns
      \begin{column}{0.32\textwidth}
        \begin{tcolorbox}[transparentStyle, title=\tColorBoxTitle{#2}]
          \oneFig{#1}{0.9}
        \end{tcolorbox}
      \end{column}
      %\begin{column}{0.05\textwidth}\end{column}
      \begin{column}{0.32\textwidth}
        \begin{tcolorbox}[transparentStyle, title=\tColorBoxTitle{#4}]
          \oneFig{#3}{0.9}
        \end{tcolorbox}
      \end{column}
      %\begin{column}{0.05\textwidth}\end{column}
      \begin{column}{0.32\textwidth}
        \begin{tcolorbox}[transparentStyle, title=\tColorBoxTitle{#6}]
          \oneFig{#5}{0.9}
        \end{tcolorbox}       
      \end{column}
  \end{columns}
}

\newcommand{\threeFigColumnsMedium}[6]{    %soti changed size - temporary
  \begin{columns}[T] %For stable vertical alignment of columns
      \begin{column}{0.32\textwidth}
        \begin{tcolorbox}[transparentStyle, title=\tColorBoxTitle{#2}]
          \oneFig{#1}{0.85}
        \end{tcolorbox}
      \end{column}
      %\begin{column}{0.05\textwidth}\end{column}
      \begin{column}{0.32\textwidth}
        \begin{tcolorbox}[transparentStyle, title=\tColorBoxTitle{#4}]
          \oneFig{#3}{0.85}
        \end{tcolorbox}
      \end{column}
      %\begin{column}{0.05\textwidth}\end{column}
      \begin{column}{0.32\textwidth}
        \begin{tcolorbox}[transparentStyle, title=\tColorBoxTitle{#6}]
          \oneFig{#5}{0.85}
        \end{tcolorbox}       
      \end{column}
  \end{columns}
}




%%% rows x columns = 1 x 3 figures (3 titles)
\newcommand{\threeFigColumnsCustomSize}[7]{
  \begin{columns}[T] %For stable vertical alignment of columns
      \begin{column}{0.32\textwidth}
        \begin{tcolorbox}[transparentStyle, title=\tColorBoxTitle{#2}]
          \oneFig{#1}{#7}
        \end{tcolorbox}
      \end{column}
      %\begin{column}{0.05\textwidth}\end{column}
      \begin{column}{0.32\textwidth}
        \begin{tcolorbox}[transparentStyle, title=\tColorBoxTitle{#4}]
          \oneFig{#3}{#7}
        \end{tcolorbox}
      \end{column}
      %\begin{column}{0.05\textwidth}\end{column}
      \begin{column}{0.32\textwidth}
        \begin{tcolorbox}[transparentStyle, title=\tColorBoxTitle{#6}]
          \oneFig{#5}{#7}
        \end{tcolorbox}       
      \end{column}
  \end{columns}
}

%%% rows x columns = 1 x 4 figures (4 titles)
\newcommand{\fourFigColumns}[8]{
  \vspace{-0.3cm}
  \begin{columns}[T] %For stable vertical alignment of columns
      \begin{column}{0.2\textwidth}
        \begin{tcolorbox}[transparentStyle, title=\tColorBoxTitle{#2}]
          \oneFig{#1}{0.90} %0.85
        \end{tcolorbox}
      \end{column}
      \begin{column}{0.2\textwidth}
        \begin{tcolorbox}[transparentStyle, title=\tColorBoxTitle{#4}]
          \oneFig{#3}{0.90}
        \end{tcolorbox}
      \end{column}
      \begin{column}{0.2\textwidth}
        \begin{tcolorbox}[transparentStyle, title=\tColorBoxTitle{#6}]
          \oneFig{#5}{0.90}
        \end{tcolorbox}
      \end{column}
      \begin{column}{0.2\textwidth}
        \begin{tcolorbox}[transparentStyle, title=\tColorBoxTitle{#8}]
          \oneFig{#7}{0.90}
        \end{tcolorbox}
      \end{column}
  \end{columns}
}

\newcommand{\fourFigColumnsS}[9][]{
 % Macro to include three figure in frame (side-by-side)
  % Each column occupies exactly 0.5 of the text width.
  % Each figure within a given columns occupies a customisable
  % fraction of the available space [0, 1]
  %
  % #1 = OPTIONAL width of both figures as a fraction of text width  
  % #2 = full path to figure on left
  % #3 = width of figure on left as a fraction of text width [0, 1]
  % #4 = full path to figure on center left
  % #5 = width of figure in center left as a fraction of text width [0, 1]
  % #6 = full path to figure on center right
  % #7 = width of figure on center right as a fraction of text width [0, 1]
  % #8 = full path to figure on right
  % #9 = width of figure on right as a fraction of text width [0, 1]

  \ifthenelse{\isempty{#1}}%
            {\def\mySize{1.0}}% if #1 is empty
            {\def\mySize{#1}}
  
  \begin{columns}[T] %For stable vertical alignment of columns
      \begin{column}{0.25\textwidth}
        \begin{tcolorbox}[transparentStyle, title=\tColorBoxTitle{#2}]
          \oneFig{#3}{\mySize}
        \end{tcolorbox}
      \end{column}
      \begin{column}{0.25\textwidth}
        \begin{tcolorbox}[transparentStyle, title=\tColorBoxTitle{#4}]
          \oneFig{#5}{\mySize}
        \end{tcolorbox}
      \end{column}
      \begin{column}{0.25\textwidth}
        \begin{tcolorbox}[transparentStyle, title=\tColorBoxTitle{#6}]
          \oneFig{#7}{\mySize}
        \end{tcolorbox}
      \end{column}
      \begin{column}{0.25\textwidth}
        \begin{tcolorbox}[transparentStyle, title=\tColorBoxTitle{#8}]
          \oneFig{#9}{\mySize}
        \end{tcolorbox}
      \end{column}
  \end{columns}
}

\newcommand{\sixFigs}[6]{
  \vspace{-0.3cm}
  \begin{columns}[T] %For stable vertical alignment of columns
      \begin{column}{0.161\textwidth}
        \begin{tcolorbox}[transparentStyle, title=\tColorBoxTitle{}]
          \oneFig{#1}{1.12} %0.85
        \end{tcolorbox}
      \end{column}
      \begin{column}{0.161\textwidth}
        \begin{tcolorbox}[transparentStyle, title=\tColorBoxTitle{}]
          \oneFig{#2}{1.12}
        \end{tcolorbox}
      \end{column}
      \begin{column}{0.161\textwidth}
        \begin{tcolorbox}[transparentStyle, title=\tColorBoxTitle{}]
          \oneFig{#3}{1.12}
        \end{tcolorbox}
      \end{column}
      \begin{column}{0.161\textwidth}
        \begin{tcolorbox}[transparentStyle, title=\tColorBoxTitle{}]
          \oneFig{#4}{1.12}
        \end{tcolorbox}
      \end{column}
      \begin{column}{0.161\textwidth}
        \begin{tcolorbox}[transparentStyle, title=\tColorBoxTitle{}]
          \oneFig{#5}{1.12}
        \end{tcolorbox}
            \end{column}
      \begin{column}{0.161\textwidth}
        \begin{tcolorbox}[transparentStyle, title=\tColorBoxTitle{}]
          \oneFig{#6}{1.12}
        \end{tcolorbox}
      \end{column}
  \end{columns}
}

%%% rows x columns = 2 x 2 figures (4 titles)
\newcommand{\fourFigColumnsRatio}[8]{
  \vspace{-0.3cm}
  \begin{columns}[T] %For stable vertical alignment of columns
      \begin{column}{0.5\textwidth}
        \begin{tcolorbox}[transparentStyle, title=\tColorBoxTitle{#2}]
          \oneFig{#1}{0.42}
        \end{tcolorbox}
      \end{column}
      \begin{column}{0.5\textwidth}
        \begin{tcolorbox}[transparentStyle, title=\tColorBoxTitle{#4}]
          \oneFig{#3}{0.42}
        \end{tcolorbox}
      \end{column}
  \end{columns}\vspace{0.2cm}
  \begin{columns}[T] %For stable vertical alignment of columns
      \begin{column}{0.5\textwidth}
        \begin{tcolorbox}[transparentStyle, title=\tColorBoxTitle{#6}]
          \oneFig{#5}{0.42}
        \end{tcolorbox}
      \end{column}
      \begin{column}{0.5\textwidth}
        \begin{tcolorbox}[transparentStyle, title=\tColorBoxTitle{#8}]
          \oneFig{#7}{0.42}
        \end{tcolorbox}
      \end{column}
  \end{columns}
}

%%% rows x columns = 3 x 3 figures (1 title) and 1 x 2 (1 title)
\newcommand{\sixFigColumns}[9]{
  % #9 typical value is 0.25
    \begin{tcolorbox}[transparentStyle, title=\tColorBoxTitle{#1}] \end{tcolorbox}
    \begin{columns}[T] %For stable vertical alignment of columns
    \begin{column}{#9\textwidth} %0.32 for plots without ration
      \begin{tcolorbox}[transparentStyle] \oneFig{#2}{1.0} \end{tcolorbox}
    \end{column}
    \begin{column}{#9\textwidth}
      \begin{tcolorbox}[transparentStyle] \oneFig{#3}{1.0} \end{tcolorbox}
    \end{column}
    \begin{column}{#9\textwidth}
      \begin{tcolorbox}[transparentStyle] \oneFig{#4}{1.0} \end{tcolorbox}
    \end{column}
    \end{columns}
    %\vspace{-0.2cm}
    \begin{tcolorbox}[transparentStyle, title=\tColorBoxTitle{#5}] \end{tcolorbox}
    \begin{columns}[T] %For stable vertical alignment of columns
    \begin{column}{#9\textwidth}
      \begin{tcolorbox}[transparentStyle] \oneFig{#6}{1.0} \end{tcolorbox}
    \end{column}
    \begin{column}{#9\textwidth}
      \begin{tcolorbox}[transparentStyle] \oneFig{#7}{1.0} \end{tcolorbox}
    \end{column}
    \begin{column}{#9\textwidth}
      \begin{tcolorbox}[transparentStyle] \oneFig{#8}{1.0} \end{tcolorbox}
    \end{column}
  \end{columns}
}

%%% rows x columns = 3 x 3 figures
\newcommand{\nineFigs}[9]{
  \vspace{-0.5cm}
  \threeFigColumnsCustomSize{empty}{}{empty}{}{#9}{#8}{0.6}
  \vspace{0.0cm}
  \threeFigColumnsCustomSize{empty}{}{empty}{}{#7}{#8}{0.6}
  \vspace{0.0cm}
  \threeFigColumnsCustomSize{#1}{#2}{#3}{#4}{#5}{#6}{0.6}
}


%%% rows x columns = 3 x 2 figures (L Shape)
\newcommand{\fiveFigsLShapeReversed}[9]{
    \begin{columns}[T] %For stable vertical alignment of columns
    \begin{column}{0.66\textwidth}
      #1
    \end{column}
    \begin{column}{0.34\textwidth}
      \twoFigColumnsVertical{#2}{#3}{#4}{#5}{0.45}
    \end{column}
    \end{columns}
    \vspace{0.2cm}
    \threeFigColumnsCustomSize{#6}{boosted top}{#7}{boosted \bW{+}}{#8}{resolved top}{0.45}
}

%%% rows x columns = 3 x 2 figures (L Shape)
\newcommand{\fiveFigsLShapeReversedAlt}[9]{
  \begin{columns}[T] %For stable vertical alignment of columns
    \begin{column}{0.7\textwidth}
      #1
    \end{column}
    \begin{column}{0.3\textwidth}
      \twoFigVertical{#2}{#3}{#4}{#5}
    \end{column}
  \end{columns}
  \begin{columns}[T] %For stable vertical alignment of columns
    \begin{column}{1.0\textwidth}
      \threeFig{#6}{#7}{#8}{#9}
    \end{column}
  \end{columns}
}

%%% rows x columns = 2 x 4 figures (1 title) and 1 x 2 (1 title)
\newcommand{\eightFigColumns}[8]{
    \begin{tcolorbox}[transparentStyle] \end{tcolorbox}
    \begin{columns}[T] %For stable vertical alignment of columns
    \begin{column}{0.24\textwidth}
      \begin{tcolorbox}[transparentStyle] \oneFig{#1}{1.0} \end{tcolorbox}
    \end{column}
    \begin{column}{0.24\textwidth}
      \begin{tcolorbox}[transparentStyle] \oneFig{#2}{1.0} \end{tcolorbox}
    \end{column}
    \begin{column}{0.24\textwidth}
      \begin{tcolorbox}[transparentStyle] \oneFig{#3}{1.0} \end{tcolorbox}
    \end{column}
    \begin{column}{0.24\textwidth}
      \begin{tcolorbox}[transparentStyle] \oneFig{#4}{1.0} \end{tcolorbox}
    \end{column}
  \end{columns}
    \begin{tcolorbox}[transparentStyle] \end{tcolorbox}
    \begin{columns}[T] %For stable vertical alignment of columns
    \begin{column}{0.24\textwidth}
      \begin{tcolorbox}[transparentStyle] \oneFig{#5}{1.0} \end{tcolorbox}
    \end{column}
    \begin{column}{0.24\textwidth}
      \begin{tcolorbox}[transparentStyle] \oneFig{#6}{1.0} \end{tcolorbox}
    \end{column}
    \begin{column}{0.24\textwidth}
      \begin{tcolorbox}[transparentStyle] \oneFig{#7}{1.0} \end{tcolorbox}
    \end{column}
    \begin{column}{0.24\textwidth}
      \begin{tcolorbox}[transparentStyle] \oneFig{#8}{1.0} \end{tcolorbox}
    \end{column}
\end{columns}
}


%%%%%%%%%%%%%%%%%%%%%%%%%%%%%%%%%%%%%%%%%%%%%%%%%%%%%%%%%%%%%%%%%%%%%%%%%%%%%%%%%%%%%%%%%%%%%%%%%%%%%%%%%
% Animations
%%%%%%%%%%%%%%%%%%%%%%%%%%%%%%%%%%%%%%%%%%%%%%%%%%%%%%%%%%%%%%%%%%%%%%%%%%%%%%%%%%%%%%%%%%%%%%%%%%%%%%%%%
% 1x1 animation
\newcommand{\oneAnimation}[4]{
  \begin{columns}[T] %For stable vertical alignment of columns
    \begin{column}{#4\textwidth}
      \begin{tcolorbox}[animationStyle]
        \begin{center}
          \animategraphics[width=1.0\textwidth, #3]{#2}{./figures/animations/#1}{}{}
        \end{center}
      \end{tcolorbox}
    \end{column}
  \end{columns}
}

% 1x1 animation, 1 title
\newcommand{\oneAnimColumn}[5]{
  \begin{columns}[T] %For stable vertical alignment of columns
    \begin{column}{#5\textwidth}
      \begin{tcolorbox}[animationStyle, title=\tColorBoxTitle{#2}]
        \begin{center}
          \animategraphics[width=1.0\textwidth, #4]{#3}{./figures/animations/#1}{}{}
          %\mediabutton[jsaction={anim.myAnim.playFwd();}]{\fbox{\strut Play}}
          %\mediabutton[jsaction={anim.myAnim.pause();}]{\fbox{\strut Pause}}
        \end{center}
      \end{tcolorbox}
    \end{column}
  \end{columns}
}

%%% 1x2 animation, 2 titles
\newcommand{\twoAnimColumns}[9]{
  \begin{columns}[T] %For stable vertical alignment of columns
      \begin{column}{#9\textwidth}
        \begin{tcolorbox}[animationStyle,title=\tColorBoxTitle{#2}]
        \begin{center}
          \animategraphics[width=1.0\textwidth, #4]{#3}{./figures/animations/#1}{}{}
        \end{center}
    \end{tcolorbox}
      \end{column}
      \begin{column}{#9\textwidth}
        \begin{tcolorbox}[animationStyle,title=\tColorBoxTitle{#6}]
        \begin{center}
          \animategraphics[width=1.0\textwidth, #8]{#7}{./figures/animations/#5}{}{}
        \end{center}
    \end{tcolorbox}       
      \end{column}
  \end{columns}
}


%%%%%%%%%%%%%%%%%%%%%%%%%%%%%%%%%%%%%%%%%%%%%%%%%%%%%%%%%%%%%%%%%%%%%%%%%%%%%%%%%%%%%%%%%%%%%%%%%%%%%%%%%
% Columns for Text & Formulae
%%%%%%%%%%%%%%%%%%%%%%%%%%%%%%%%%%%%%%%%%%%%%%%%%%%%%%%%%%%%%%%%%%%%%%%%%%%%%%%%%%%%%%%%%%%%%%%%%%%%%%%%%
\newcommand{\twoTitledColumns}[4]{
\begin{columns}[T] %For stable vertical alignment of columns
  \begin{column}{0.50\textwidth}
    \centering\SpecialColour{\bf{#1}}\\\vspace{0.17cm}
    \scriptsize{#2}
  \end{column}
  \begin{column}{0.50\textwidth}
    \centering\SpecialColour{\bf{#3}}\\
    \scriptsize{#4}
  \end{column}
\end{columns}
}

%%% 1x2
\newcommand{\twoColumns}[2]{
\begin{columns}[T] %For stable vertical alignment of columns
  \begin{column}{0.49\textwidth}
    \scriptsize{#1}
    %\tiny{#1}
  \end{column}
  \begin{column}{0.02\textwidth} \end{column}
  \begin{column}{0.49\textwidth}
    \scriptsize{#2}
    %\tiny{#2}
  \end{column}
\end{columns}
}

%%% 1x2
\newcommand{\twoColumnsAsym}[2]{
\begin{columns}[T] %For stable vertical alignment of columns
  \begin{column}{0.69\textwidth}
    \scriptsize{#1}
    %\tiny{#1}
  \end{column}
  \begin{column}{0.02\textwidth} \end{column}
  \begin{column}{0.29\textwidth}
    \scriptsize{#2}
    %\tiny{#2}
  \end{column}
\end{columns}
}

%%% 1x2
\newcommand{\twoColumnsAsymOp}[2]{
\begin{columns}[T] %For stable vertical alignment of columns
  \begin{column}{0.31\textwidth}
    \scriptsize{#1}
    %\tiny{#1}
  \end{column}
  \begin{column}{0.1\textwidth} \end{column}
  \begin{column}{0.59\textwidth}
    \scriptsize{#2}
    %\tiny{#2}
  \end{column}
\end{columns}
}

%%% 1x2
\newcommand{\twoColumnsAsymAlt}[2]{
\begin{columns}[T] %For stable vertical alignment of columns
  \begin{column}{0.55\textwidth}
    \scriptsize{#1}
    %\tiny{#1}
  \end{column}
  \begin{column}{0.02\textwidth} \end{column}
  \begin{column}{0.43\textwidth}
    \scriptsize{#2}
    %\tiny{#2}
  \end{column}
\end{columns}
}


%%% 1x3
\newcommand{\threeColumns}[3]{
\begin{columns}[T] %For stable vertical alignment of columns
  \begin{column}{0.3333\textwidth}
    \small
        {#1}
  \end{column}
  \begin{column}{0.3333\textwidth}
    \small
        {#2}
  \end{column}
  \begin{column}{0.3333\textwidth}
    \small
        {#3}
  \end{column}
\end{columns}
\normalsize
}

%%% 1x2, align env
\newcommand{\twoColumnsAlign}[2]{
  \begin{columns}[T] %For stable vertical alignment of columns[totalwidth=1.0\textwidth]
    \begin{column}{0.5\textwidth}
      \begin{align*}
        #1
      \end{align*}
    \end{column}
    \begin{column}{0.5\textwidth}
      \begin{align*}
  	#2
      \end{align*}
    \end{column}
  \end{columns}
}

%%% 1x2, itemize env
\newcommand{\twoColumnsItemize}[2]{
\begin{columns}[T] %For stable vertical alignment of columns
  \begin{column}{0.5\textwidth}
    \begin{itemize}
      %\scriptsize{#1}
      #1
    \end{itemize}    
  \end{column}
  \begin{column}{0.5\textwidth}
    \begin{itemize}
      %\scriptsize{#2}
      #2
    \end{itemize}
  \end{column}
\end{columns}
\normalsize
}

%%% 1x2, itemize env
\newcommand{\threeColumnsItemize}[3]{
\begin{columns}[T] %For stable vertical alignment of columns
  \begin{column}{0.33\textwidth}
    \begin{itemize}
      \scriptsize{#1}
    \end{itemize}    
  \end{column}
  \begin{column}{0.33\textwidth}
    \begin{itemize}
      \scriptsize{#2}
    \end{itemize}
  \end{column}
  \begin{column}{0.33\textwidth}
    \begin{itemize}
      \scriptsize{#3}
    \end{itemize}
  \end{column}
\end{columns}
\normalsize
}

%%% 1x4, itemize env
\newcommand{\fourColumnsItemize}[4]{
\begin{columns}[T] %For stable vertical alignment of columns
  \begin{column}{0.25\textwidth}
    \begin{itemize}
      \scriptsize{#1}
    \end{itemize}    
  \end{column}
  \begin{column}{0.25\textwidth}
    \begin{itemize}
      \scriptsize{#2}
    \end{itemize}
  \end{column}
  \begin{column}{0.25\textwidth}
    \begin{itemize}
      \scriptsize{#3}
    \end{itemize}
  \end{column}
  \begin{column}{0.25\textwidth}
    \begin{itemize}
      \scriptsize{#4}
    \end{itemize}
  \end{column}
\end{columns}
\normalsize
}

%%%%%%%%%%%%%%%%%%%%%%%%%%%%%%%%%%%%%%%%%%%%%%%%%%%%%%%%%%%%%%%%%%%%%%%%%%%%%%%%%%%%%%%%%%%%%%%%%%%%%%%%%
% General Shortcuts
%%%%%%%%%%%%%%%%%%%%%%%%%%%%%%%%%%%%%%%%%%%%%%%%%%%%%%%%%%%%%%%%%%%%%%%%%%%%%%%%%%%%%%%%%%%%%%%%%%%%%%%%%
\newcommand{\GoToBackup}[1]{\hyperlink{#1}{\nameref{backup:#1} \refBackup{#1}}}
\newcommand{\GoToBackupNoPage}[1]{\hyperlink{#1}{\nameref{backup:#1}}}
%\newcommand{\GoToBackupCustom}[2]{\hyperlink{#1}{\nameref{backup:#2}}}
%\newcommand{\RefBackupSlide}[2]{\MyHRef{#1}{#2}}
\newcommand{\RefBackupSlide}[2]{\hyperlink{#1}{#2}}
\newcommand{\GoTo}[1]{\hyperlink{#1}{\nameref{sec:#1}}}
\newcommand{\LinkToFrameGo}[2]{\hyperlink{#1}{\beamergotobutton{#2}} }
\newcommand{\LinkToFrameBack}[2]{\hyperlink{#1}{\beamerreturnbutton{#2}} }
\newcommand{\BackToNavigation}{\vfill \hfill \LinkToFrameBack{Backup}{Navigation}}
\newcommand{\hilight}[2]{\colorbox{#1}{#2}}
\newcommand{\Show}[1]{\temporal<#1>{\color{gray}}{\color{black}}{\color{black}}}
\newcommand{\ShowCustom}[2]{\temporal<#1>{\color{gray}}{\color{#2}}{\color{#2}}}
\newcommand{\SmallNote}[1]{\Emph{(#1)}}%\normalsize)} %\scriptsize\Emph{#1}s\normalsize)}
\newcommand{\TinyNote}[1]{(\tiny\Emph{#1}\normalsize)}
\newcommand{\SpecialColour}[1]{\textcolor{\MySectionColour}{#1}}
\newcommand{\SpecialColourAlt}[1]{\textcolor{kGreen}{#1}}
\newcommand{\Emph}[1]{\SpecialColour{#1}}
\newcommand{\TextColour}[2]{\textcolor{#1}{#2}}
\newcommand{\NewLineCentre}[1]{\begin{center} #1 \end{center}}
\newcommand{\refslide}[1]{Slide~\ref{Slide:#1}\xspace}
\newcommand{\refBackup}[1]{(page \ref{#1})\xspace}
\newcommand{\CloudText}[1]{
  \begin{tikzpicture}
    \node[align=center,kGreen!50, draw,fill, text=white, cloud, cloud puffs=12, aspect=3.0, scale=0.5] at (0,0) {\textbf{\Large \bf{#1}}};
\end{tikzpicture}
}
\newcommand{\Cloud}[3][180]% [angle], content
           {%
             \begin{tikzpicture}[overlay]
               \node[align=center, fill=#2, draw,
                 text=\MySubsectionTextColour, cloud callout,
                 cloud puffs=17, cloud puff arc=140,
                 callout pointer segments=3, anchor=pointer, callout
                 relative pointer={(#1:2 cm )}, aspect=4,scale=0.5] at
               (0.2ex,0.5ex) {\sc{#3} };
             \end{tikzpicture}
           }

\newcommand{\Note}[4][180]% [angle], content
           {%
             \begin{tikzpicture}[overlay]
               %\node[align=center, fill=#2, draw,
                 %text=\MySubsectionTextColour,
                 %ellipse callout,
               %rectangle callout,
               \node[align=center,
                 fill=white,
                 text=#2,
                 draw=#2,
                 thick,
                 inner sep=0.2cm,
                 #4, %ellipse callout, rectangle callout
                 %callout pointer segments=3,
                 anchor=pointer,
                 callout relative pointer={(#1:0.5 cm )}, % distance to callout 
                 aspect=1, scale=0.65] at (0.2ex, 0.6ex) {\MakeUppercase{\bf{#3}} };
             \end{tikzpicture}
           }

\newcommand{\NoteAlt}[4][180]% [angle], content
           {%
             \begin{tikzpicture}[overlay]
               \node[align=center, fill=white, text=#2, draw=#2, thick,
                 inner sep=0.2cm,
                 #4, %ellipse callout, rectangle callout
                 %callout pointer segments=3,
                 anchor=pointer,
                 callout relative pointer={(#1:0.5 cm )}, % distance to callout 
                 aspect=1, scale=0.65] at (0.2ex, 0.6ex) {\bf{#3}};
             \end{tikzpicture}
           }
           
\newcommand{\SampleTag}[2]{\Note[180]{#1}{#2}{rectangle callout}}
\newcommand{\ByATLASTag}{\Note[180]{kHPlusBlue}{ATLAS}{rectangle callout}}
\newcommand{\CiteTag}[1]{\NoteAlt[180]{kHPlusBlue}{#1}{rectangle callout}}
\newcommand{\DoneTag}{\Note[180]{kDarkGreen}{Done}{rectangle callout}}
\newcommand{\PreliminaryTag}{\Note[180]{kHLTausRed}{Preliminary!}{rectangle callout}}
\newcommand{\vPreliminaryTag}{\Note[180]{kHLTausRed}{Very Preliminary}{rectangle callout}}
\newcommand{\ValidateTag}{\Note[180]{kHLTausOrange}{Validate}{rectangle callout}}
\newcommand{\BugTag}{\Note[180]{kHLTausRed}{Bug}{rectangle callout}}
\newcommand{\DebugTag}{\Note[180]{kHLTausRed}{Debug}{rectangle callout}}
\newcommand{\NewTag}{\Note[180]{kDarkGreen}{New}{rectangle callout}}
\newcommand{\NewNextTag}{\Note[180]{kHLTausRed}{New (Next Slide)}{rectangle callout}}
\newcommand{\OldTag}{\Note[180]{kHLTausOrange}{Old}{rectangle callout}}
%\newcommand{\DevTag}{\Note[180]{kHLTausBlue}{Dev}{rectangle callout}}
\newcommand{\DevTag}{\Note[180]{kHLTausBlue}{In Development}{rectangle callout}}
\newcommand{\ProgressTag}{\Note[180]{kHLTausOrange}{In Progress}{rectangle callout}}
\newcommand{\Callout}[1]{\Note[180]{kDarkGreen}{#1}{rectangle callout}}
\newcommand{\SomeTag}[2]{\Note[180]{#2}{#1}{rectangle callout}}
\newcommand{\MCTag}{\Note[180]{kHLTausRed}{MC}{rectangle callout}}
\newcommand{\MCMatchedTag}{\Note[180]{kHLTausRed}{MC-Matched}{rectangle callout}}
\newcommand{\QCDTag}{\Note[180]{QCD}{QCD=Data-EWK}{rectangle callout}}
\newcommand{\FromDataTag}{\Note[180]{kHLTausRed}{From Data}{rectangle callout}}
\newcommand{\FromMCTag}{\Note[180]{kHLTausBlue}{From MC}{rectangle callout}}
\newcommand{\TodoListTag}{\Note[180]{kHLTausBlue}{To-Do
    List}{rectangle callout}}
\newcommand{\OrthogonalityTag}{\Note[180]{kGreen}{Orthogonality}{rectangle callout}}
\newcommand{\TTMCTag}{\Note[180]{kDarkGreen}{\normalsize \lowercase{$t\bar{t}$} MC}{rectangle callout}}
\newcommand{\QCDMCTag}{\Note[180]{kHLTausOrange}{\normalsize QCD MC}{rectangle callout}}
\newcommand{\PtRhoTag}{\Note[180]{kDarkGreen}{\normalsize\lowercase{$p_{T}$, $\rho$ dependent}}{rectangle callout}}
\newcommand{\DashedLineTag}{\Note[180]{blue}{\normalsize\lowercase{Dashed line}}{rectangle callout}}
\newcommand{\SolidLineTag}{\Note[180]{blue}{\normalsize\lowercase{Solid line}}{rectangle callout}}
\newcommand{\TodoTag}{\Note[180]{kHLTausBlue}{To-Do}{rectangle callout}}
\newcommand{\RequestedTag}{\Note[180]{kDarkGreen}{Requested}{rectangle
    callout}}
\newcommand{\ThisTalkTag}{\Note[180]{kPurple}{This Talk}{rectangle callout}}
%=================================================
% Soti's Tags
%=================================================

\newcommand{\GenuineTag}{\Note[180]{kDarkGreen}{Genuine}{rectangle callout}}
\newcommand{\FakeTag}{\Note[180]{kHLTausRed}{Fake}{rectangle callout}}
\newcommand{\ResolvedTag}{\Note[180]{kDarkGreen}{Resolved}{rectangle callout}}
\newcommand{\BoostedTag}{\Note[180]{kBlue}{Boosted}{rectangle callout}}
\newcommand{\TopDirectionGenuineTag}{\Note[180]{kHPlusBlue}{Top Direction Genuine}{rectangle callout}}
\newcommand{\TopDirectionFakeTag}{\Note[180]{kHLTausOrange}{Top Direction Fake}{rectangle callout}}

\newcommand{\GouskosANTag}{\Note[180]{kHLTausBlue}{AN 2016/437 - \textit{Gouskos et al.}}{rectangle callout}}
\newcommand{\TTnote}{\Note[180]{kHLTausRed}{TT}{rectangle callout}}
\newcommand{\QCDnote}{\Note[180]{kHLTausBlue}{QCD}{rectangle callout}}

\newcommand{\BDTGenuineTag}{\Note[180]{kDarkGreen}{BDT Genuine}{rectangle callout}}

\newcommand{\NormToLumi}{\Note[180]{kHLTausOrange}{Normalized to Lumi}{rectangle callout}}
\newcommand{\NormToOne}{\Note[180]{kHLTausRed}{Normalized to Unity}{rectangle callout}}

\newcommand{\pTorder}{\Note[180]{kHLTausRed}{\pT order}{rectangle
    callout}}
\newcommand{\BDTorder}{\Note[180]{kDarkGreen}{BDT order}{rectangle callout}}

\newcommand{\Investigate}{\Note[180]{kDarkGreen}{Investigate}{rectangle callout}}
\newcommand{\nextSlide}{\Note[180]{kHLTausRed}{Next Slide}{rectangle callout}}
\newcommand{\InvestigateTag}{\Note[180]{kHLTausBlue}{Investigate}{rectangle
    callout}}

\newcommand{\SimulationTag}{\Note[180]{kDarkGreen}{Simulation}{rectangle callout}}
\newcommand{\DataDrivenTag}{\Note[180]{kRed}{Data driven}{rectangle callout}}
\newcommand{\FixmeTag}{\Note[180]{red}{FIXME}{rectangle callout}}

\NewDocumentCommand\StickyNote{O{6cm}mO{6cm}}{%
  \begin{tikzpicture}
    \node[
      drop shadow={
        shadow xshift=2pt,
        shadow yshift=-4pt
      },
      inner xsep=7pt,
      fill=kPostitYellow,
      xslant=-0.1,
      yslant=0.1,
      inner ysep=10pt
    ] {\parbox[t][#1][c]{#3}{\texttt{#2}}};
  \end{tikzpicture}%
}

%%%%%%%%%%%%%%%%%%%%%%%%%%%%%%%%%%%%%%%%%%%%%%%%%%%%%%%%%%%%%%%%%%%%%%%%%%%%%%%%%%%%%%%%%%%%%%%%%%%%%%%%%
% Special characters and icons
%%%%%%%%%%%%%%%%%%%%%%%%%%%%%%%%%%%%%%%%%%%%%%%%%%%%%%%%%%%%%%%%%%%%%%%%%%%%%%%%%%%%%%%%%%%%%%%%%%%%%%%%%
\newcommand{\CloudTextAlt}[1]{
  \begin{tikzpicture}
    \node[align=center,kRed!50, draw,fill, text=white, cloud, cloud puffs=12, aspect=3.0, scale=0.5] at (0,0) {\textbf{\Large \bf{#1}}};
\end{tikzpicture}
}
\newcommand{\EllipseText}[1]{
  \begin{tikzpicture}
    \node[align=center,kGreen!50, draw,fill, text=white, ellipse, cloud puffs=12, aspect=3.0, scale=0.5] at (0,0) {\textbf{\Large \bf{#1}}};
\end{tikzpicture}
}
\newcommand{\EllipseTextAlt}[1]{
  \begin{tikzpicture}
    \node[align=center,kRed!50, draw,fill, text=white, ellipse, cloud puffs=12, aspect=3.0, scale=0.5] at (0,0) {\textbf{\Large \bf{#1}}};
\end{tikzpicture}
}
\newcommand{\PolygonText}[1]{
  \begin{tikzpicture}
    \node[align=center,kGreen!50, draw,fill, text=white, regular polygon,regular polygon sides=5, aspect=3.0, scale=0.5] at (0,0) {\textbf{\Large \bf{#1}}};
\end{tikzpicture}
}
\newcommand{\StarText}[1]{
  \begin{tikzpicture}
    \node[align=center,kGreen!50, draw,fill, text=white, star, star points=6, scale=0.5] at (0,0) {\textbf{\Large \bf{#1}}};
\end{tikzpicture}
}

\newcommand{\Winner}[1]{#1~\textcolor{kGreen}{\ding{51}}} %\checkmark}}
\newcommand{\Loser}[1]{#1~\textcolor{kRed}{\ding{55}}}
\newcommand{\Good}{\textcolor{kGreen}{\ding{51}}} %https://tex.stackexchange.com/questions/42619/x-mark-to-match-checkmark
\newcommand{\Bad}{\textcolor{kRed}{\ding{55}}}
%https://tex.stackexchange.com/questions/42619/x-mark-to-match-checkmark
\newcommand{\BadBlack}{\textcolor{kBlack}{\ding{55}}}
\newcommand{\SoSo}{\textcolor{kOrange}{\ding{51}}} %https://tex.stackexchange.com/questions/42619/x-mark-to-match-checkmark

%%%%%%%%%%%%%%%%%%%%%%%%%%%%%%%%%%%%%%%%%%%%%%%%%%%%%%%%%%%%%%%%%%%%%%%%%%%%%%%%%%%%%%%%%%%%%%%%%%%%%%%%%
% Slide Definitions
%%%%%%%%%%%%%%%%%%%%%%%%%%%%%%%%%%%%%%%%%%%%%%%%%%%%%%%%%%%%%%%%%%%%%%%%%%%%%%%%%%%%%%%%%%%%%%%%%%%%%%%%%
% Slide Transitions:
% Horizontal blinds pulled away: \transblindshorizontal
% Vertical blinds pulled away: \transblindsvertical
% Move to center from all sides: \transboxin
% Move to all sides from center: \transboxout
% Slowly dissolve what was shown before: \transdissolve
% Glitter sweeps in specified direction: \transglitter
% Sweeps two vertical lines in:   \transslipverticalin
% Sweeps two vertical lines out:  \transslipverticalout
% Sweeps two horizontal lines in: \transhorizontalin
% Sweeps two horizontal lines out: \transhorizontalout
% Sweeps single line in specified direction: \transwipe
% Show slide specified number of seconds: \transduration{2}
% New (not documented yet): \transcover, \transfade, \transpush, \transuncover
%%%%%%%%%%%%%%%%%%%%%%%%%%%%%%%%%%%%%%%%%%%%%%%%%%%%%%%%%%%%%%%%%%%%%%%%%%%%%%%%%%%%%%%%%%%%%%%%%%%%%%%%%
\newcommand{\BeginLabelledSlide}[3]{
\begin{frame}[#2]\vspace{-0.2cm} %http://tex.stackexchange.com/questions/116390/beamer-frame-environment-introduces-5-69pt-vertical-margin
  \begin{ChangeMargin}{+0.2cm}{+0.2cm}
    \label{#1} 
    \setcounter{subfigure}{0}% Reset subfigure counter (otherwise it keeps counting)
    #3
    %\BackToNavigation % not really needed as I have the navigation on top!
  \end{ChangeMargin}
\end{frame}
}

\newcommand{\BackupSlide}[4]{
  \MySubSectionNoTOC{#1}
  \BeginLabelledSlide{#2}{#3}{#4}
}

% http://tex.stackexchange.com/questions/110229/beamer-full-width-image-aligned-to-top-with-a-frame-title
\newcommand{\Slide}[2]{
  \begin{frame}[#1] \vspace{-0.2cm} %http://tex.stackexchange.com/questions/116390/beamer-frame-environment-introduces-5-69pt-vertical-margin   
    \begin{ChangeMargin}{+0.2cm}{+0.2cm}
      %%% Slide transitions only work in full-view-mode (adobe acrobat)
      %\transfade
      %\transwipe
      \setcounter{subfigure}{0}% Reset subfigure counter (otherwise it keeps counting)
      #2
    \end{ChangeMargin}
  \end{frame}
}

%%%%%%%%%%%%%%%%%%%%%%%%%%%%%%%%%%%%%%%%%%%%%%%%%%%%%%%%%%%%%%%%%%%%%%%%%%%%%%%%%%%%%%%%%%%%%%%%%%%%%%%%%
% Environment Definitions
%%%%%%%%%%%%%%%%%%%%%%%%%%%%%%%%%%%%%%%%%%%%%%%%%%%%%%%%%%%%%%%%%%%%%%%%%%%%%%%%%%%%%%%%%%%%%%%%%%%%%%%%%
% See: http://stackoverflow.com/questions/1670463/latex-change-margins-of-only-a-few-pages
\newenvironment{ChangeMargin}[2]{%
\begin{list}{}{%
\setlength{\topsep}{0cm}%
\setlength{\leftmargin}{#1}%
\setlength{\rightmargin}{#2}%
\setlength{\listparindent}{\parindent}%
\setlength{\itemindent}{\parindent}%
\setlength{\parsep}{\parskip}%
}%
\item[]}{\end{list}}


% See: http://tex.stackexchange.com/questions/167711/showing-margins-frames-in-beamer-slides
\newenvironment{CustomiseLayout}[9]{%
  \setlength{\hoffset}{#1}%
  \setlength{\oddsidemargin}{#2}%
  \setlength{\headheight}{#3}%
  \setlength{\textheight}{#4}%
  \setlength{\textwidth}{#5}%
  \setlength{\footskip}{#6}%
  \setlength{\voffset}{#7}%
  \setlength{\topmargin}{#8}%
  \setlength{\headsep}{#9}%
  \setlength{\marginparsep}{0cm}%
  \setlength{\marginparwidth}{0.0cm}%
  \setlength{\marginparpush}{0cm}%
  \setlength{\voffset}{0cm}%  
}%

%%%%%%%%%%%%%%%%%%%%%%%%%%%%%%%%%%%%%%%%%%%%%%%%%%%%%%%%%%%%%%%%%%%%%%%%%%%%%%%%%%%%%%%%%%%%%%%%%%%%%%%%%
% Custom Section Types
%%%%%%%%%%%%%%%%%%%%%%%%%%%%%%%%%%%%%%%%%%%%%%%%%%%%%%%%%%%%%%%%%%%%%%%%%%%%%%%%%%%%%%%%%%%%%%%%%%%%%%%%%
\newcommand{\MySectionWithFile}[2]{
  \section{#1}{
    \label{sec:#1}
    \input{./tex/#2}
  }
}

\newcommand{\MySectionWithFileNoTOC}[2]{
  \section*{#1}{
    \label{sec:#1}
    \input{./tex/#2}
  }
}

\newcommand{\MySection}[1]{
  \section{#1}
  \label{sec:#1}
}

\newcommand{\MySectionNoTOC}[1]{
  \section*{#1}
  \label{sec:#1}
}

\newcommand{\MySubSectionBegin}[3]{
  \AtBeginSubsection[]
  {
    \Slide{c}{
      \frametitle{#3}
      \oneFigNoLabel{#1}{#2}
    }
  }
}

\newcommand{\DisableAtBeginSubsection}{\AtBeginSubsection[]{}}


\newcommand{\MySubSection}[1]{
  \subsection{#1}
  %\label{subsec:#1} %not needed at the moment. it gives warning when  two subsections have the same title
}

\newcommand{\MySubSectionNoTOC}[1]{
  \subsection*{#1}{}
  \label{subsec:#1}
}

\newcommand{\MyBackupSubSectionNoTOC}[1]{
  \subsection*{#1}{}
  \label{backup:#1}
}

\newcommand{\MyBackupSection}[2]{
  \MyBackupSubSectionNoTOC{#1}{}
  \BeginLabelledSlide{#1}{t}{#2}
  %\BeginLabelledSlide{backup:#1}{t}{#2\BackToNavigation}
}


%%%%%%%%%%%%%%%%%%%%%%%%%%%%%%%%%%%%%%%%%%%%%%%%%%%%%%%%%%%%%%%%%%%%%%%%%%%%%%%%%%%%%%%%%%%%%%%%%%%%%%%%%
% Boxes for Formulae, Statements, Laws
%%%%%%%%%%%%%%%%%%%%%%%%%%%%%%%%%%%%%%%%%%%%%%%%%%%%%%%%%%%%%%%%%%%%%%%%%%%%%%%%%%%%%%%%%%%%%%%%%%%%%%%%%
% WARNING: \CustomiseTextMargins{0.0cm}{0.0cm} in talk.tex must be changed to add margins!
\newcommand{\PhysQuote}[1]{\begin{quote}\centering{\color{\MySectionColour}\Ovalbox{\color{\MySectionColour}#1}}\end{quote}}
%
\newcommand{\PhysDefinition}[2]{
  \SetDefaultRoundBoxColour
  \begin{beamerboxesrounded}[upper=uppercol,lower=lowercol,shadow=false]{#1} #2 \end{beamerboxesrounded}
  \SetDefaultRoundBoxColour
}
%
\newcommand{\PhysLaw}[2]{
 \setbeamercolor{uppercol}{fg=white, bg=kGreen}
  \setbeamercolor{lowercol}{fg=black, bg=kGreen!20}
  \begin{beamerboxesrounded}[upper=uppercol,lower=lowercol,shadow=false]{#1} #2 \end{beamerboxesrounded}
  \SetDefaultRoundBoxColour
}
%
\newcommand{\PhysFormula}[2]{
  \setbeamercolor{uppercol}{fg=white, bg=kOrange}
  \setbeamercolor{lowercol}{fg=black, bg=kOrange!20}
  \begin{beamerboxesrounded}[upper=uppercol,lower=lowercol,shadow=false]{#1} \vspace{-0.4cm}\begin{align*}#2\end{align*} \end{beamerboxesrounded}
  \SetDefaultRoundBoxColour
}
%
\newcommand{\PhysMultipleChoiceQuestion}[3]{
  \setbeamercolor{uppercol}{fg=white, bg=\MySectionColour}
  \setbeamercolor{lowercol}{fg=black, bg=\MySectionColour!20}
  \begin{beamerboxesrounded}[upper=uppercol,lower=lowercol,shadow=false]{Ερώτηση}\begin{enumerate}#1\end{enumerate} \only<2>{\begin{align*}#2\end{align*}}\end{beamerboxesrounded}
  \SetDefaultRoundBoxColour
}
%
\newif\ifExamples\Examplesfalse
\ifExamples\else\newcounter{examples} \setcounter{examples}{0}\Examplestrue\fi 
%
\newcommand{\PhysNewExample}[1]{
  \addtocounter{examples}{1}
  \setbeamercolor{uppercol}{fg=white, bg=\MySectionColour}
  \setbeamercolor{lowercol}{fg=black, bg=\MySectionColour!20}
  \begin{beamerboxesrounded}[upper=uppercol,lower=lowercol,shadow=false]{Example \theexamples} #1 \end{beamerboxesrounded}
  \SetDefaultRoundBoxColour
}
%
\newcommand{\PhysExample}[1]{
  \setbeamercolor{uppercol}{fg=white, bg=\MySectionColour}
  \setbeamercolor{lowercol}{fg=black, bg=\MySectionColour!20}
  \begin{beamerboxesrounded}[upper=uppercol,lower=lowercol,shadow=false]{Example \theexamples} #1 \end{beamerboxesrounded}
  \SetDefaultRoundBoxColour
}

%
\newcommand{\PhysExampleNoCount}[1]{
  \setbeamercolor{uppercol}{fg=white, bg=\MySectionColour}
  \setbeamercolor{lowercol}{fg=black, bg=\MySectionColour!20}
  \begin{beamerboxesrounded}[upper=uppercol,lower=lowercol,shadow=false]{Example} #1 \end{beamerboxesrounded}
  \SetDefaultRoundBoxColour
}
%
\newcommand{\PhysHomework}[2]{
  \setbeamercolor{uppercol}{fg=white, bg=kRed}
  \setbeamercolor{lowercol}{fg=black, bg=kRed!20}
  \begin{beamerboxesrounded}[upper=uppercol,lower=lowercol,shadow=false]{Homework} #1 \only<2> {#2}\end{beamerboxesrounded}
  \SetDefaultRoundBoxColour
}
%
\newcommand{\PhysWorkedExample}[3]{
  \setbeamercolor{uppercol}{fg=white, bg=kRed}
  \setbeamercolor{lowercol}{fg=black, bg=kRed!20}
  \begin{beamerboxesrounded}[upper=uppercol,lower=lowercol,shadow=false]{\scriptsize#1}  \scriptsize#2 \only<2> {\scriptsize#3}\end{beamerboxesrounded}
  \SetDefaultRoundBoxColour
}
%
\newcommand{\PhysWorkedExampleCont}[3]{
  \setbeamercolor{uppercol}{fg=white, bg=kRed}
  \setbeamercolor{lowercol}{fg=black, bg=kRed!20}
  \begin{beamerboxesrounded}[upper=uppercol,lower=lowercol,shadow=false]{\scriptsize#1}  \scriptsize#2 {\scriptsize#3}\end{beamerboxesrounded}
  \SetDefaultRoundBoxColour
}
%
\newcommand{\SetDefaultRoundBoxColour}{
  \setbeamercolor{uppercol}{fg=white, bg=kBlue}
  \setbeamercolor{lowercol}{fg=black, bg=kBlue!20}
}
%
\newcommand{\BoxEq}[2]{
 \setbeamercolor{uppercol}{fg=white, bg=#1}
  \setbeamercolor{lowercol}{fg=black, bg=#1!20}
  \begin{beamerboxesrounded}[upper=uppercol,lower=lowercol,shadow=false]{}#2\end{beamerboxesrounded}
  \SetDefaultRoundBoxColour
}

%\newcommand{\MyHRef}[3]{\setbeamercolor{alerted text}{fg=black,bg=#3}\alert{\href{#1}{#2}} }
%\newcommand{\MyHRef}[2]{\setbeamercolor{alerted text}{fg=\MySectionColour,bg=white}\alert{\href{#1}{#2}}}
\newcommand{\MyHRef}[2]{\setbeamercolor{alerted text}{fg=\MySectionColour,bg=white}\alert{\href{#1}{\textbf{#2}}}}
\newcommand{\HtmlLink}[1]{\href{#1}{\HighLight{website}}\xspace}
\newcommand{\EqLine}[1]{ \visible<+->{#1}\\}
\newcommand{\Eq}[1]{\EqLine{#1}}
\newcommand{\EEq}[1]{\EEqLine{#1}}
\newcommand{\EEqLine}[1]{\visible<+->{#1}}
\newcommand{\TextLine}[1]{\visible<+->{#1}}
\newcommand<>{\mycolorbox}[3][white]{{%
    % #1 = default color when overlay specification is not fulfilled (default: white)
    % #2 = color when overlay specification is fulfilled
    % #3 = text to be displayed
    % #4 = the actual overlay specification
    \def\mycolor{#1}%
    \only#4{\def\mycolor{#2}}%
    \colorbox{\mycolor}{#3}%
}}


%%%%%%%%%%%%%%%%%%%%%%%%%%%%%%%%%%%%%%%%%%%%%%%%%%%%%%%%%%%%%%%%%%%%%%%%%%%%%%%%%%%%%%%%%%%%%%%%%%%%%%%%%
% Enumerate Customisations (TikZ)
%%%%%%%%%%%%%%%%%%%%%%%%%%%%%%%%%%%%%%%%%%%%%%%%%%%%%%%%%%%%%%%%%%%%%%%%%%%%%%%%%%%%%%%%%%%%%%%%%%%%%%%%
\newcommand*\circled[1]{\tikz[baseline=(char.base)]{ \node[circle,ball color=kBlue§, shade,  color=white,inner sep=1.2pt] (char) {\tiny #1};}}
\newcommand*\rounded[1]{\tikz[baseline=(char.base)]{ \node[draw=none,ball color=kBlue, shade,  color=white, rounded corners=3.5pt, inner sep=2.5pt] (char) {\scriptsize #1};}}
\newcommand*{\ballref}[1]{%
    \begin{pgfpicture}{-1ex}{-0.65ex}{1ex}{1ex}
    \usebeamercolor[fg]{item projected}
    {\pgftransformscale{1.75}\pgftext{\normalsize\pgfuseshading{bigsphere}}}
    {\pgftransformshift{\pgfpoint{0pt}{0.5pt}}
      \pgftext{\usebeamerfont*{item projected}\ref{#1}}}
\end{pgfpicture}}%

%%%%%%%%%%%%%%%%%%%%%%%%%%%%%%%%%%%%%%%%%%%%%%%%%%%%%%%%%%%%%%%%%%%%%%%%%%%%%%%%%%%%%%%%%%%%%%%%%%%%%%%%%
% Phase-2 (TTI WG) Shortcuts
%%%%%%%%%%%%%%%%%%%%%%%%%%%%%%%%%%%%%%%%%%%%%%%%%%%%%%%%%%%%%%%%%%%%%%%%%%%%%%%%%%%%%%%%%%%%%%%%%%%%%%%%%
\newcommand{\CaloTau}[1]{\TextColour{kBlack}{#1}}
\newcommand{\TkTauFromCalo}[1]{\TextColour{kRed}{#1}}
\newcommand{\TkEmTau}[1]{\TextColour{kGreen}{#1}}

\newcommand{\Calo}[1]{\TextColour{kRed}{#1}}
\newcommand{\Tk}[1]{\TextColour{kvLightBlue}{#1}}
\newcommand{\VtxIso}[1]{\TextColour{kGreen}{#1}}
\newcommand{\RelIso}[1]{\TextColour{kPurple}{#1}}
\newcommand{\RateRef}[1]{\TextColour{kGray}{#1}}
\newcommand{\threshCalo}[1]{\TextColour{kRed}{$\eT = $ #1~$\GeV$}}
\newcommand{\threshTk}[1]{\TextColour{kvLightBlue}{$\eT = $ #1~$\GeV$}}
\newcommand{\threshVtxIso}[1]{\TextColour{kGreen}{$\eT = $ #1~$\GeV$}}
\newcommand{\threshRelIso}[1]{\TextColour{kPurple}{$\eT = $ #1~$\GeV$}}
\newcommand{\Vals}[3]{ \textcolor{kRed}{#1} \,\, \textcolor{kGreen}{#2} \,\, \textcolor{kBlue}{#3} }
\newcommand{\Highlight}[1]{\tikz\node[rectangle, fill=kOrange, inner sep=0.2mm]{#1};}

\newcommand{\SSModules}[1]{\TextColour{kRed}{#1}}
\newcommand{\PSModules}[1]{\TextColour{kBlue}{#1}}
\newcommand{\PixModules}[1]{\TextColour{kPurple}{#1}}

\newcommand{\HitPatternSeven}{\textcolor{red}{L1 + L2 + L3}}
\newcommand{\HitPatternEleven}{\textcolor{violet}{L1 + L2 + D1}}
\newcommand{\HitPatternThirteen}{\textcolor{pink}{L1 + L3 + D1}}
\newcommand{\HitPatternFourteen}{\textcolor{black}{L2 + L3 + D1}}
\newcommand{\HitPatternNineteen}{\textcolor{teal}{L1 + L2 + D1}}
\newcommand{\HitPatternFifteen}{\textcolor{orange}{L1 + L2 + L3 + D1}}
\newcommand{\HitPatternTwentryThree}{\textcolor{blue}{L1 + L2 + L3 + D2}}
\newcommand{\HitPatternFiftyOne}{\textcolor{green}{L1 + L2 + D2 + D3}}





\def\overlaynode<#1>#2;{
        \gdef\stacknodecommonpart{#2}
        \pgfkeys{/typeset node/.list={#1}}
}

\pgfkeys{
    /typeset node/.code={
        \edef\pgf@marshal{\noexpand\node[#1]\stacknodecommonpart;}
        \pgf@marshal
    }
}

\newcommand\PlaceText[3]{%
\begin{tikzpicture}[remember picture,overlay]
\node[outer sep=0pt,inner sep=0pt,anchor=south west] 
  at ([xshift=#1,yshift=-#2]current page.north west) {#3};
\end{tikzpicture}%
}
